\setcounter{page}{1} % Iniciar la numeración de las páginas en este punto.
\pagestyle{fancy}

\section{Resumen}

El consumo eléctrico residencial suele gestionarse de manera reactiva debido a la falta de herramientas que permitan desglosar y analizar el gasto energético en tiempo real. Este Proyecto presenta el diseño e implementación de un sistema de gestión inteligente basado en el Internet de las Cosas (IoT), cuyo propósito es dotar al usuario de capacidades de monitoreo granular y control activo sobre su instalación.

La solución técnica consiste en una arquitectura distribuida de nodos de adquisición independientes, basados en microcontroladores ESP32, que permiten la medición desagregada de corriente por habitaciones o circuitos específicos para la detección de consumo fantasma. La comunicación del sistema se fundamenta en el protocolo MQTT, centralizando el flujo de información en un servidor local desplegado en una Raspberry Pi, donde se integran servicios de lógica de control (Node-RED), bases de datos de series temporales (InfluxDB) y visualización web (Grafana). Adicionalmente, el sistema cierra el lazo de gestión mediante la integración de actuadores WiFi (Smart Switches) y algoritmos de automatización, permitiendo la ejecución de horarios programados y el corte remoto del suministro, validando así una alternativa accesible y privada para la eficiencia energética en el hogar.
\newpage

\section{Introducción}
    En el contexto global actual, la eficiencia energética se ha consolidado como un pilar fundamental para el desarrollo sostenible, impulsada tanto por la necesidad de mitigar el impacto ambiental como por la optimización de los recursos económicos \cite{chevez2018energias}. El sector residencial representa una porción significativa del consumo total de energía eléctrica, sin embargo, la mayoría de los usuarios carecen de herramientas que les permitan comprender y gestionar su consumo de manera efectiva. La facturación eléctrica tradicional ofrece únicamente un resumen mensual, funcionando como un indicador tardío que no permite identificar patrones de uso específicos ni detectar fuentes de desperdicio, como el conocido consumo fantasma o en modo de espera (standby) \cite{iea2022standby}. La problemática central que se aborda es la falta de acceso a información granular y en tiempo real sobre el consumo de energía en el hogar, así como la ausencia de mecanismos accesibles para actuar sobre dicho consumo en caso de anomalías o necesidades de gestión remota. Esta carencia impide a los usuarios tomar decisiones informadas y acciones directas para optimizar el uso de sus aparatos y reducir su gasto energético. Para dar solución a esta necesidad, se propone el diseño y la implementación de un \textbf{sistema inteligente de monitoreo y control energético residencial}, basado en hardware de bajo costo y software de código abierto. El desarrollo del proyecto se fundamentará en la integración de varias tecnologías clave. En primer lugar, el \textbf{sensado de corriente no invasivo} \cite{fraden2016handbook}, que utiliza un transformador de corriente (SCT-013) para medir el flujo eléctrico de forma segura. En segundo lugar, se implementará una arquitectura de \textbf{medición distribuida} mediante una red de \textbf{nodos embebidos} (ESP32) \cite{vahid2002embedded}. Cada nodo actuará como punto de adquisición independiente en distintas zonas o habitaciones, procesando la señal de su respectivo sensor y transmitiendo los datos simultáneamente a través del protocolo de mensajería \textbf{MQTT}, estándar en el Internet de las Cosas (IoT) por su eficiencia. Finalmente, se configurará un \textbf{servidor local en una Raspberry Pi} que centralizará la gestión del sistema: el broker \textbf{EMQX} recibirá los datos, \textbf{Node-RED} orquestará el flujo de información para su procesamiento, detección de anomalías y gestión de horarios automatizados, enviando \textbf{notificaciones vía Telegram} cuando sea necesario. Una base de datos \textbf{InfluxDB} almacenará las series temporales de consumo, y \textbf{Grafana} proporcionará una interfaz web alojada en la misma \textbf{Raspberry Pi} para la \textbf{visualización} de los datos en tiempo real e históricos. Este sistema no solo proporcionará una medición precisa y desagregada del consumo, sino que permitirá la \textbf{detección inteligente de anomalías} y ofrecerá al usuario la capacidad de \textbf{controlar remotamente el suministro eléctrico} por zonas mediante \textbf{actuadores inteligentes WiFi}, añadiendo una capa de gestión activa, automatización y seguridad al monitoreo energético del hogar. \newpage

\section{Antecedentes}

    El monitoreo del consumo de energía eléctrica mediante tecnologías del IoT ha sido abordado en diversos trabajos académicos a nivel nacional e internacional. Para contextualizar la presente propuesta y destacar su originalidad, se analizan a continuación cinco trabajos relevantes que representan el estado actual de la tecnología. \subsection{Trabajo de Grado CUC: Diseño de un prototipo de un sistema domótico} Este trabajo describe el diseño de un sistema domótico que incluye la monitorización del consumo eléctrico en una vivienda. El prototipo utiliza sensores conectados a un microcontrolador y visualiza los datos a través de una aplicación móvil desarrollada específicamente para el proyecto. \cite{AhumadaBarros2021} 

        \paragraph{Semejanzas:} Comparte el enfoque residencial y el uso de una interfaz gráfica para que el usuario visualice su consumo.
        \paragraph{Diferencias y Aportación:} El trabajo se centra en la medición y visualización mediante una app a medida.
        Nuestra propuesta se diferencia al implementar una arquitectura de servidor local (Raspberry Pi) con un stack de software industrial (Grafana/InfluxDB) y añade capacidades de control activo (corte de suministro) y notificaciones por Telegram, ausentes en este antecedente.
    \subsection{Paper IEEE ROPEC 2021: Smart IoT Device For Energy Consumption Monitoring}
    Este trabajo propone un dispositivo IoT basado en ESP8266 que transmite datos a un servidor web (PHP/MySQL).
    Su característica principal es el uso de redes neuronales para identificar qué electrodoméstico específico está consumiendo energía (técnica NILM).
    \cite{LopezAlfaro2021}

        \paragraph{Semejanzas:} Uso de microcontroladores ESP y monitoreo en tiempo real vía WiFi.
        \paragraph{Diferencias y Aportación:} Este paper se enfoca en la identificación de cargas mediante IA.
        Nuestro proyecto prioriza la gestión y seguridad: detección de anomalías de consumo general y la capacidad de actuación remota (corte de energía).
        Además, utilizamos una base de datos de series de tiempo (InfluxDB), más eficiente que la base SQL del paper.
    \subsection{Paper IEEE ROPEC 2019: Energy Monitoring Consumption at IoT-Edge}
    Este trabajo propone un circuito de monitoreo con sensor SCT-013 y ESP8266.
    Implementa una red neuronal en el propio microcontrolador para detectar consumos inusuales, enviando los datos a una base de datos MySQL en una Raspberry Pi.
    \cite{AguirreNunez2019}

        \paragraph{Semejanzas:} Coincide plenamente en el hardware (SCT-013, ESP, RPi) y el objetivo de detectar anomalías.
        \paragraph{Diferencias y Aportación:} La diferencia es el método de detección y almacenamiento. El paper usa IA embebida y MySQL.
        Nosotros delegamos la lógica a Node-RED en el servidor (facilitando la configuración de reglas) y usamos InfluxDB.
        Aportamos la capacidad crítica de control remoto que este trabajo no contempla.
    \subsection{Trabajo Final UNLP: Diseño e Implementación de un Sistema de Medición...}
    Este proyecto detalla un sistema de medición residencial utilizando hardware de código abierto.
    Los datos son procesados para su presentación al usuario mediante una interfaz web local.
    \cite{Cicutti2019}

        \paragraph{Semejanzas:} Uso de hardware accesible y filosofía de bajo costo.
        \paragraph{Diferencias y Aportación:} Nuestra propuesta expande este concepto integrando la orquestación de flujos de datos con Node-RED y un sistema de alertas proactivas.
        A diferencia de la visualización pasiva de este trabajo, nuestro sistema permite al usuario intervenir físicamente en la instalación a través de actuadores distribuidos.
    \subsection{Proyecto de Código Abierto: OpenEnergyMonitor}
    Plataforma de referencia mundial en hardware y software abierto.
    Permite monitorear consumo, generación y controlar cargas mediante su software EmonCMS.
    \cite{openenergymonitor}

        \paragraph{Semejanzas:} Base de inspiración para el uso de sensores no invasivos y la filosofía de código abierto.
        \paragraph{Diferencias y Aportación:} OpenEnergyMonitor es un ecosistema complejo y modular.
        Nuestra aportación es la integración de una solución 'todo en uno' más accesible, utilizando herramientas genéricas y populares (Telegram, Grafana) en lugar de un CMS especializado.
    \newpage 

    \subsection{Comparación Cualitativa de Trabajos Relacionados}
    
     La Tabla \ref{tab:comparacion_antecedentes} resume las similitudes y diferencias clave entre los trabajos analizados y la propuesta presentada, destacando las contribuciones originales de este proyecto en el contexto del monitoreo y control energético residencial.
    \begin{table}[H] 
        \centering
        \caption{Comparación cualitativa de los trabajos relacionados.}
        \label{tab:comparacion_antecedentes}
        \small 
        \resizebox{\textwidth}{!}{%
        \begin{tabular}{>{\centering\arraybackslash}p{2cm} p{5.5cm} p{5.5cm}} 
        \toprule
        \textbf{Ref.} & \textbf{Similitudes} & \textbf{Diferencias} \\
        \midrule

        \cite{AhumadaBarros2021} & 
   
        \begin{itemize}[leftmargin=*, topsep=0pt, itemsep=0pt, parsep=0pt, partopsep=0pt]
            \item Enfoque residencial.
            \item Interfaz en App móvil.
        \end{itemize} & 
        \begin{itemize}[leftmargin=*, topsep=0pt, itemsep=0pt, parsep=0pt, partopsep=0pt]
            \item Software a medida vs. Stack estándar.
            \item Sin control activo.
        \end{itemize} \\
        \midrule

        \cite{LopezAlfaro2021} & 
        \begin{itemize}[leftmargin=*, topsep=0pt, itemsep=0pt, parsep=0pt, partopsep=0pt]
            \item Uso de ESP y WiFi.
            \item Monitoreo tiempo real.
        \end{itemize} & 
        \begin{itemize}[leftmargin=*, topsep=0pt, itemsep=0pt, parsep=0pt, partopsep=0pt]
            \item Enfoque en IA (Identificación) vs. Control/Seguridad.
            \item SQL vs. InfluxDB.
        \end{itemize} \\
        \midrule

        \cite{AguirreNunez2019} & 
        \begin{itemize}[leftmargin=*, topsep=0pt, itemsep=0pt, parsep=0pt, partopsep=0pt]
            \item Hardware idéntico (SCT, ESP, RPi).
            \item Detección anomalías.
        \end{itemize} & 
        \begin{itemize}[leftmargin=*, topsep=0pt, itemsep=0pt, parsep=0pt, partopsep=0pt]
            \item IA embebida vs. Lógica Node-RED.
            \item Solo monitoreo vs. Monitoreo + Control.
        \end{itemize} \\
        \midrule

        \cite{Cicutti2019} & 
        \begin{itemize}[leftmargin=*, topsep=0pt, itemsep=0pt, parsep=0pt, partopsep=0pt]
            \item Hardware abierto.
            \item Visualización web local.
        \end{itemize} & 
        \begin{itemize}[leftmargin=*, topsep=0pt, itemsep=0pt, parsep=0pt, partopsep=0pt]
            \item Visualización pasiva vs. Gestión activa.
            \item Sin notificaciones.
        \end{itemize} \\
        \midrule

        \cite{openenergymonitor} & 
        \begin{itemize}[leftmargin=*, topsep=0pt, itemsep=0pt, parsep=0pt, partopsep=0pt]
            \item Sensado no invasivo.
            \item Plataforma auto-alojable.
        \end{itemize} & 
        \begin{itemize}[leftmargin=*, topsep=0pt, itemsep=0pt, parsep=0pt, partopsep=0pt]
            \item Complejo/modular vs. Integrado.
            \item EmonCMS vs. Stack MQTT/Node-RED.
        \end{itemize} \\

        \bottomrule
        \end{tabular}
        }
    \end{table}
\newpage
    
\section{Justificación}

    El análisis de las opciones actuales para el monitoreo energético residencial revela una limitación significativa: la mayoría de las herramientas se restringen a una visualización pasiva de datos o presentan barreras de alto costo y complejidad técnica.
    Esta carencia deja al usuario promedio sin capacidad de respuesta ante anomalías o desperdicios en su consumo.
    La propuesta justifica su realización al cubrir este vacío mediante un sistema que trasciende la simple observación para integrar el control activo.
    A diferencia de los medidores convencionales, este proyecto cierra el lazo de gestión permitiendo al usuario no solo ver su consumo, sino actuar sobre él mediante el corte remoto del suministro y la recepción de alertas automáticas.
    La implementación de una arquitectura basada en un servidor local es clave para resolver los problemas de privacidad y dependencia de servicios externos, garantizando que los datos sensibles permanezcan bajo el control del usuario.
    Desde la perspectiva de la ingeniería, el proyecto demuestra la viabilidad de democratizar tecnologías avanzadas de gestión energética utilizando hardware accesible y software de código abierto.
    Esta solución equilibra la robustez técnica con la economía, entregando a los hogares una herramienta efectiva de seguridad y eficiencia que supera las limitaciones funcionales de los sistemas puramente informativos.
\newpage

\section{Objetivos}

    \subsection{Objetivo General}
        Diseñar e Implementar un sistema electrónico inteligente para el monitoreo y control del consumo energético residencial, para proporcionar al usuario una herramienta accesible de gestión remota del suministro eléctrico.
    \subsection{Objetivos Particulares}
    \begin{enumerate}
        \item \textbf{Diseñar} el circuito electrónico de acondicionamiento para la señal del sensor de corriente no invasivo SCT-013, asegurando que la salida de voltaje se mantenga dentro del rango operativo de 0V a 3.3V del convertidor analógico-digital (ADC) del ESP32.
        \item \textbf{Construir} los nodos de medición distribuidos, ensamblando para cada uno su respectivo sensor SCT-013, el circuito de acondicionamiento y el microcontrolador ESP32, permitiendo así la monitorización independiente de dos zonas (habitaciones) en el prototipo.
        \item \textbf{Diseñar e implementar} el firmware para el ESP32 que adquiera las lecturas del sensor, calcule la potencia eléctrica instantánea y publique los resultados vía MQTT al servidor local.
        \item \textbf{Establecer} el servidor local en la Raspberry Pi, instalando y configurando el broker MQTT, la base de datos InfluxDB y el sistema de visualización Grafana.
        \item \textbf{Programar} en Node-RED los flujos de trabajo para procesar los datos MQTT, almacenarlos en InfluxDB, detectar condiciones de consumo anómalo y ejecutar rutinas de automatización, incluyendo la programación de horarios de corte y reconexión del suministro.
        \item \textbf{Implementar} el sistema de notificación de alertas por consumo anómalo mediante una comunicación con Telegram controlado desde Node-RED.
        \item \textbf{Integrar} actuadores inteligentes WiFi al sistema, configurando su comunicación MQTT con el servidor local para permitir el corte de suministro remoto, ya sea por alertas de seguridad, comandos manuales o itinerarios programados automáticamente.
        \item \textbf{Verificar} que la precisión del sistema de medición se encuentre dentro de un margen de error del $\pm 5\%$ en comparación con un wattmetro comercial, bajo diferentes perfiles de carga residencial.
        \item \textbf{Validar} la funcionalidad completa del sistema, verificando la correcta operación de la visualización en Grafana, las notificaciones en Telegram y la activación inalámbrica de los actuadores.
    \end{enumerate}
\newpage

\section{Marco Teórico}

    El desarrollo de sistemas de monitoreo energético requiere una comprensión profunda de la interacción entre la teoría de circuitos de corriente alterna, la instrumentación electrónica y las arquitecturas de comunicación en red.

    \subsection{La corriente alterna}

        La corriente alterna (AC) es el tipo de corriente eléctrica predominante en las redes de distribución residencial a nivel mundial. Se dice que, a diferencia de la corriente continua (DC), donde el flujo de electrones es unidireccional, en la corriente alterna la magnitud y dirección varían periódicamente siguiendo una forma de onda sinusoidal \cite{sadiku2013}. Como se aprecia en la \textbf{Figura \ref{fig:AC}}, esta oscilación característica permite una transmisión más eficiente de la energía y responde a la naturaleza mecánica de los generadores rotativos.

        En el contexto nacional, la Norma Oficial Mexicana NOM-001-SEDE-2012 establece que la red eléctrica residencial suministra una tensión nominal de 127V a 60Hz \cite{nom2012}. Esta frecuencia implica que la señal completa se repite 60 veces por segundo, resultando en un periodo temporal de $T \approx 16.67$ milisegundos por ciclo.

        La naturaleza variable de la corriente alterna introduce complejidades en la medición, ya que voltaje y corriente son funciones dependientes del tiempo $v(t)$ e $i(t)$, lo que requiere técnicas de muestreo digital para su correcto análisis \cite{boylestad2011}.

        \begin{figure}[H]
            \centering
            \includegraphics[width=1\textwidth]{img/AC.png}
            \caption[Representación de una señal de corriente alterna]{Representación de una señal de corriente alterna\footnotemark}
            \label{fig:AC}
        \end{figure}
        \footnotetext{Fuente: Elaboración propia utilizando GeoGebra.}

    \subsection{Fundamentos de Medición Eléctrica}

        Para la gestión eficiente de energía, es imperativo comprender la dinámica de las variables físicas \cite{boylestad2011}. Como se observa en la \textbf{Figura \ref{fig:AC_amplitud}}, la amplitud define los valores máximos y mínimos de la señal, mientras que la frecuencia determina la velocidad de oscilación.

        \begin{figure}[H]
            \centering
            \includegraphics[width=1\textwidth]{img/AC_amplitud.png}
            \caption[Amplitud de una señal de corriente alterna]{Amplitud de una señal de corriente alterna\footnotemark}
            \label{fig:AC_amplitud}
        \end{figure}
        \footnotetext{Fuente: Elaboración propia utilizando GeoGebra y Paint.}

        \subsubsection{Potencia Instantánea}

            La potencia instantánea se define como la tasa de transferencia de energía en un momento específico. Matemáticamente, corresponde al producto de la tensión $v(t)$ y la corriente $i(t)$ en dicho instante. En un circuito de corriente alterna esta magnitud no es estática, sino que fluctúa en el tiempo \cite{hayt2012}, tal como se expresa en la Ecuación \ref{eq:potencia_t}:

            \begin{equation}
                p(t) = v(t) \cdot i(t)
                \label{eq:potencia_t}
            \end{equation}

            Esta relación es fundamental porque proporciona la base para el cálculo posterior de la potencia promedio y el consumo energético real.

            Dado que el sistema propuesto utiliza procesamiento digital en un microcontrolador ESP32, la señal continua debe ser discretizada para su análisis. Asi que, la potencia instantánea para la muestra $n$-ésima se calcula mediante el producto de las muestras digitales \cite{openenergymonitor}, como se muestra en la Ecuación \ref{eq:potencia_n}:

            \begin{equation}
                p[n] = v[n] \cdot i[n]
                \label{eq:potencia_n}
            \end{equation}

            \vspace{0.3cm}
            \noindent \textit{\textbf{Nota técnica:} En implementaciones donde no se cuenta con un sensor de voltaje y se asume una tensión fija (ej. 127V), el término $v[n]$ en la Ecuación \ref{eq:potencia_n} se sustituye por una constante $V_{rms}$. Sin embargo, la definición teórica rigurosa para el cálculo de potencia real requiere el muestreo simultáneo de ambas señales.}

        \subsubsection{Potencia Promedio (Potencia Activa)}

            Si bien la potencia instantánea describe el flujo de energía en cada momento, para propósitos de facturación y diseño de máquinas es necesario conocer el promedio de esta energía a lo largo del tiempo. Este valor se denomina Potencia Promedio o Potencia Activa ($P$).

            Matemáticamente, la potencia promedio se define como la integral de la potencia instantánea $p(t)$ sobre un periodo completo $T$, dividida por la duración de dicho periodo \cite{hayt2012}, como se muestra en la ecucación \ref{eq:potencia_activa_t}
        
            \begin{equation}
                P = \frac{1}{T} \int_{0}^{T} p(t) \, dt = \frac{1}{T} \int_{0}^{T} v(t) i(t) \, dt
                \label{eq:potencia_activa_t}
            \end{equation}

            Físicamente, la Potencia Activa representa la cantidad de energía eléctrica que es irreversiblemente transformada en otra forma de energía (mecánica, térmica o lumínica) por unidad de tiempo. Su unidad es el Watt ($W$).

            Para la implementación digital en el microcontrolador, el cálculo difiere del utilizado para la Potencia Aparente. La Potencia Activa real requiere integrar muestra por muestra antes de promediar para capturar correctamente el desfase entre voltaje y corriente. En el firmware del sistema propuesto, se aplica la sumatoria discreta documentada en las notas de aplicación de monitoreo energético \cite{programarfacil}, tal como se expresa en la Ecuación \ref{eq:potencia_activa_n}:

            \begin{equation}
                P = \frac{1}{N} \sum_{n=0}^{N-1} (v[n] \cdot i[n])
                \label{eq:potencia_activa_n}
            \end{equation}

            Donde $v[n]$ e $i[n]$ son las muestras instantáneas emparejadas. Este método permite medir el consumo real ($Watts$) independientemente del factor de potencia de la carga, asegurando una lectura fiel a lo que registra un medidor comercial.

        \subsubsection{Valor Eficaz (RMS)}

            En el análisis de circuitos de corriente alterna, la magnitud de la corriente varía instante a instante, lo que dificulta establecer un valor único que represente su capacidad de trabajo. Para resolver esto se utiliza el Valor Eficaz o RMS (del inglés \textit{Root Mean Square}).

            Físicamente, el valor RMS se define como el valor equivalente de una corriente directa (DC) que produciría la misma disipación de potencia promedio (calor) al circular por una resistencia óhmica idéntica \cite{hayt2012}. Matemáticamente, para una función continua en el tiempo $i(t)$, este valor se obtiene mediante la raíz cuadrada del promedio del cuadrado de la señal a lo largo de un periodo $T$, tal como se expresa en la Ecuación \ref{eq:rms_integral}:

            \begin{equation}
                I_{RMS} = \sqrt{\frac{1}{T} \int_{0}^{T} i^2(t) \, dt}
                \label{eq:rms_integral}
            \end{equation}

            Esta definición es relevante porque vincula una señal oscilatoria con un valor escalar de potencia real, permitiendo comparar sistemas de AC y DC bajo un mismo estándar energético.

            Para la implementación del sistema de monitoreo, no es viable aplicar un promedio aritmético simple a la señal capturada. La naturaleza sinusoidal de la corriente alterna presenta simetría entre sus semiciclos positivos y negativos \cite{programarfacil}. En consecuencia, la suma directa de todos los valores instantáneos durante un ciclo completo resultaría matemáticamente en cero, lo cual no refleja el flujo real de energía.

            Para solucionar esta limitante física y obtener una medición útil, el firmware del sistema implementa el algoritmo de \textbf{Raíz Cuadrática Media} en el dominio discreto. Este proceso consta de tres etapas secuenciales que justifican la fórmula utilizada:

            \begin{enumerate}
                \item \textbf{Cuadrar (\textit{Square}):} Se eleva al cuadrado cada muestra digital $i[n]$. Esta operación matemática tiene el efecto físico de convertir todos los valores negativos de la onda en positivos, representando la magnitud de la energía sin importar la dirección del flujo.
                \item \textbf{Promediar (\textit{Mean}):} Se calcula la media aritmética de estos valores cuadráticos acumulados durante el periodo de muestreo.
                \item \textbf{Raíz (\textit{Root}):} Se aplica la raíz cuadrada al resultado para devolver la magnitud a su unidad lineal original (Amperios).
            \end{enumerate}

            Por lo tanto, la ecuación que rige el funcionamiento del algoritmo en el microcontrolador, sustituyendo la integral continua por una sumatoria de $N$ muestras, se define en la Ecuación \ref{eq:rms_discreto}:

            \begin{equation}
                I_{RMS} = \sqrt{\frac{1}{N} \sum_{n=0}^{N-1} i[n]^2}
                \label{eq:rms_discreto}
            \end{equation}

            Este método de cálculo, conocido técnicamente como "True RMS", garantiza que el sistema reporte el consumo energético real incluso ante la presencia de cargas no lineales que distorsionen la forma de onda perfecta.

        \subsubsection{Potencia Aparente ($S$)}

            Mientras que la potencia activa representa el trabajo útil, en el análisis de sistemas eléctricos es necesario cuantificar la magnitud total de la potencia que circula por el circuito, independientemente de si esta se consume o se almacena temporalmente en campos electromagnéticos. Este parámetro se denomina Potencia Aparente.

            La Potencia Aparente ($S$) se define como el producto de los valores eficaces (RMS) de voltaje y corriente \cite{hayt2012}. El término "aparente" deriva del hecho de que, si se miden el voltaje y la corriente por separado con un multímetro convencional, el producto de estas lecturas parece ser la potencia total del sistema.

            Matemáticamente, esta relación se expresa en la Ecuación \ref{eq:potencia_aparente}:

            \begin{equation}
                S = V_{RMS} \cdot I_{RMS}
                \label{eq:potencia_aparente}
            \end{equation}

            Su unidad de medida es el Volt-Amperio ($VA$), distinción necesaria para diferenciarla de la potencia activa ($Watts$).

            Desde la perspectiva de la teoría de circuitos, la potencia aparente corresponde a la magnitud vectorial de la potencia compleja. Esta magnitud relaciona la potencia activa ($P$) y la potencia reactiva ($Q$) mediante el teorema de Pitágoras, conformando lo que se conoce como el triángulo de potencias, tal como se muestra en la Ecuación \ref{eq:triangulo_potencias}:

            \begin{equation}
                S = \sqrt{P^2 + Q^2}
                \label{eq:triangulo_potencias}
            \end{equation}

            Para el prototipo de monitoreo energético desarrollado, el cálculo de $S$ es una operación aritmética directa derivada de los bloques de procesamiento anteriores. Una vez que el microcontrolador ha obtenido el voltaje RMS (ya sea medido o definido como constante nominal de 127V) y la corriente RMS (calculada mediante el algoritmo de suma de cuadrados del sensor SCT-013), el firmware realiza el producto de ambas magnitudes.

            Esta métrica es crítica para el proyecto porque representa la carga total que la instalación eléctrica debe soportar. Un valor de potencia aparente significativamente mayor que la potencia activa indica un bajo factor de potencia, lo que implica ineficiencias en el consumo energético de la vivienda.

        \subsubsection{Factor de Potencia y Eficiencia Energética}
            La relevancia del \textbf{Factor de Potencia ($FP$)} trasciende la teoría de circuitos, impactando directamente en la economía de la distribución eléctrica. La eficiencia en la transferencia de energía determina los costos de infraestructura: una instalación con un bajo factor de potencia exige a la compañía suministradora una mayor capacidad de corriente para entregar la misma cantidad de potencia útil \cite{hayt2012}. En consecuencia, los consumidores que presentan una baja eficiencia de transmisión suelen incurrir en tarifas más elevadas por kilowatt-hora ($kWh$) para compensar la inversión adicional en equipos de distribución.

            Para cuantificar este fenómeno, se define el Factor de Potencia como la razón entre la \textbf{Potencia Promedio ($P$)}, que realiza el trabajo real, y la \textbf{Potencia Aparente ($S$)}, que dimensiona la capacidad total del sistema.

            Matemáticamente, considerando un sistema lineal donde la tensión $v(t)$ tiene un ángulo de fase $\theta$ y la corriente $i(t)$ un ángulo $\phi$:

            \begin{equation}
                v(t) = V_m \cos(\omega t + \theta)
                \label{eq:tension_fase}
            \end{equation}
            \begin{equation}
                i(t) = I_m \cos(\omega t + \phi)
                \label{eq:corriente_fase}
            \end{equation}

            El factor de potencia se calcula como el coseno de la diferencia angular entre ambas señales, tal como se expresa en la Ecuación \ref{eq:fp_analitico}:

            \begin{equation}
                FP = \frac{P}{S} = \cos(\theta - \phi)
                \label{eq:fp_analitico}
            \end{equation}

            El signo y magnitud del ángulo $(\theta - \phi)$ permiten clasificar el comportamiento de la carga conectada a la red. En el caso de \textbf{cargas resistivas}, la tensión y la corriente oscilan al unísono (en fase), resultando en un factor de potencia unitario donde toda la potencia aparente es igual a la potencia útil. Por el contrario, en \textbf{cargas inductivas}, comunes en motores y transformadores, la corriente alcanza su pico después que el voltaje, generando un ángulo de fase positivo conocido como \textbf{retraso}. Finalmente, en \textbf{cargas capacitivas}, la corriente se anticipa al voltaje, generando un \textbf{adelanto} de fase.

            Para fines de este proyecto, el microcontrolador calcula la relación $P/S$ directamente de las muestras acumuladas, lo que permite diagnosticar si el consumo doméstico es eficiente o si existen cargas inductivas predominantes que degradan la red.
        \newpage

    \subsection{Sensor SCT-013}

        La medición de corriente se realiza mediante el método no invasivo utilizando sensores de la serie \textbf{SCT-013}, los cuales operan bajo el principio de los \textbf{Transformadores de Corriente (CT)}. En esta configuración, el conductor de la instalación eléctrica actúa como el devanado primario (una sola espira), mientras que el sensor integra un núcleo de ferrita partido y un devanado secundario con un número elevado de vueltas (típicamente 2000), lo que establece la relación de transformación de corriente \cite{naylamp2023sct}. Como se muestra en la \textbf{Figura \ref{fig:sct_diagrama}}.

        \begin{figure}[H]
            \centering
            \includegraphics[width=0.8\textwidth]{img/sct_diagrama.png}
            \caption[Diagrama de funcionamiento del sensor SCT-013]{Diagrama de funcionamiento del sensor SCT-013\footnotemark}
            \label{fig:sct_diagrama}
        \end{figure}
        \footnotetext{Fuente: Elaboración propia utilizando paint.}    


        Físicamente, la corriente que circula por el primario induce un flujo magnético en el núcleo, el cual genera una corriente proporcionalmente menor en el secundario. Existen dos variantes principales en esta familia de sensores:

        \begin{itemize}
            \item \textbf{Salida de Corriente (SCT-013-000):} Entrega una señal de corriente pura (ej. 50 mA por cada 100 A). Requiere una resistencia de carga externa (\textit{Burden Resistor}) para convertir esta corriente en una señal de voltaje legible.
            \item \textbf{Salida de Voltaje (SCT-013-030):} Incluye una resistencia de carga interna, entregando directamente una señal de voltaje (ej. 1 V por cada 30 A).
        \end{itemize}

        Dado que la salida del sensor es una señal de corriente alterna (AC) que oscila entre valores positivos y negativos, y considerando que el convertidor analógico-digital (ADC) del ESP32 opera únicamente con voltajes positivos (0V a 3.3V), es necesario implementar un circuito de acondicionamiento de señal.

        Este circuito añade un voltaje de desplazamiento (\textit{DC Offset}) a la señal, típicamente de $V_{CC}/2$ (1.65V para el ESP32), centrando la onda sinusoidal en el rango dinámico del ADC. Como advierte Naylamp Mechatronics \cite{naylamp2023sct}, sin este acondicionamiento los semiciclos negativos de la corriente alterna serían recortados (clipping) o podrían dañar irreversiblemente la entrada del microcontrolador.

        Para el modelo \textbf{SCT-013-000} seleccionado para este proyecto (100A), la resistencia \textit{burden} ($R_{carga}$) necesaria para obtener un voltaje pico deseado ($V_{pico}$) se calcula mediante la Ley de Ohm aplicada al secundario, tal como se muestra en la Ecuación \ref{eq:burden_resistor}:

        \begin{equation}
            R_{carga} = \frac{V_{pico}}{I_{secundario}}
            \label{eq:burden_resistor}
        \end{equation}

        Donde $I_{secundario}$ corresponde a la corriente de salida nominal del sensor (0.05 A).
        

    \subsection{Hardware del Sistema}
        \textbf{Microcontrolador ESP32:} Se selecciona este SoC (System on Chip) por su capacidad de procesamiento dual-core y conectividad WiFi integrada. Su convertidor analógico-digital (ADC) de 12 bits permite una resolución suficiente para digitalizar la señal proveniente de los sensores de corriente.
        
        \textbf{Raspberry Pi:} Actúa como el servidor de borde (\textit{Edge Server}). Es una computadora de placa reducida basada en Linux capaz de ejecutar el stack de software del servidor (Broker, Base de Datos y Dashboard) las 24 horas con un consumo energético mínimo.
        
        \textbf{Actuadores Inteligentes (Smart Switches):} Dispositivos basados en microcontroladores (tipo Sonoff Basic R4) que integran un relevador electromecánico y conectividad WiFi. Su función es interrumpir físicamente el circuito eléctrico al recibir un comando digital, permitiendo el control ON/OFF remoto.

        \subsection{Plataformas de Software}
        \textbf{Node-RED:} Herramienta de programación visual basada en flujos. Actúa como el orquestador lógico del sistema, procesando los mensajes MQTT, gestionando las reglas de automatización (horarios) y enrutando los datos.
        
        \textbf{InfluxDB:} Base de datos de series temporales (TSDB) optimizada para almacenar datos con marcas de tiempo precisas, ideal para registrar el historial de consumo eléctrico.
        
        \textbf{Grafana:} Plataforma de visualización que permite crear tableros de control (\textit{dashboards}) interactivos para que el usuario final monitoree las variables eléctricas y el estado de los actuadores en tiempo real.
\newpage

\section{Metodología - Descripción Técnica}

    El sistema evoluciona hacia una arquitectura IoT distribuida para el monitoreo y control del consumo eléctrico residencial. A diferencia de los sistemas centralizados que miden únicamente la acometida principal, esta propuesta implementa una topología de \textbf{nodos distribuidos por habitación}.
    El prototipo se compone de \textbf{dos nodos de adquisición independientes} basados en microcontroladores ESP32 y un servidor local centralizado en una Raspberry Pi. Esta configuración permite desagregar el consumo por zonas, facilitando la identificación puntual de cargas fantasma que quedarían enmascaradas en una medición general.
    El servidor centraliza el procesamiento, almacenamiento, visualización y control del sistema. La comunicación entre los nodos de adquisición, los actuadores y el servidor se realiza mediante el protocolo de mensajería MQTT a través de una red WiFi local.

    \subsection{Diagrama a Bloques del Sistema}
        La arquitectura del sistema propuesto se divide en tres subsistemas funcionales principales, como se ilustra en el diagrama a bloques de la Figura \ref{fig:diagrama_bloques}.
        Estos subsistemas son: 1) Los Nodos de Medición y Adquisición, 2) El Servidor Local de Procesamiento, y 3) El Subsistema de Control.
        \begin{figure}[H]
            \centering
            \includegraphics[width=1.1\textwidth]{img/diagrama_bloques.png}
            \caption{Diagrama a bloques del sistema propuesto (Representación de la arquitectura general).}
            \label{fig:diagrama_bloques}
        \end{figure}

         \newpage

    \subsection{Descripción de Subsistemas}
        \paragraph{1. Subsistema de Medición Distribuido (Nodos ESP32):}
        Es el componente encargado de la adquisición de datos en el borde. Para este prototipo, se implementan \textbf{dos nodos idénticos}, cada uno asignado a un circuito o habitación específica. Cada nodo opera de forma autónoma y consta de:
        \begin{itemize}
            \item \textbf{Sensor (SCT-013):} Es un transformador de corriente no invasivo de núcleo partido.
            Su función es medir la corriente alterna (AC) que fluye por el conductor de la zona asignada sin contacto eléctrico.
            \begin{itemize}
                    \item \textit{Entrada:} Corriente AC (0-100A) de la línea de fase.
                    \item \textit{Salida:} Una señal de corriente AC pequeña, proporcional a la entrada (ej. 50mA).
            \end{itemize}
            \item \textbf{Acondicionamiento de Señal:} Este circuito convierte la señal de corriente AC del sensor en una señal de voltaje DC con un desfase (offset), adaptada para ser leída por el ESP32 (ej. 0-3.3V).
            \begin{itemize}
                    \item \textit{Entrada:} Señal de corriente AC del sensor.
                    \item \textit{Salida:} Señal de voltaje DC (0-3.3V) al ADC del ESP32.
            \end{itemize}
            \item \textbf{Procesamiento (ESP32):} El microcontrolador ESP32 muestrea la señal del ADC, realiza los cálculos para obtener la potencia instantánea (W) y la potencia aparente (VA), y se conecta a la red WiFi local.
            \begin{itemize}
                    \item \textit{Entrada:} Señal de voltaje del acondicionador.
                    \item \textit{Salida:} Paquetes de datos MQTT (Protocolo: MQTT v3.1.1).
                \end{itemize}
        \end{itemize}

        \paragraph{2. Subsistema Servidor Local (Raspberry Pi):}
        Es el cerebro del sistema.
        Centraliza la recepción, procesamiento y almacenamiento de datos.
        \begin{itemize}
            \item \textbf{Broker MQTT:} Se utilizará un broker (EMQX) que actúa como intermediario de mensajería.
            Recibe los datos publicados por los nodos distribuidos.
            \item \textbf{Node-RED:} Es la herramienta de programación visual que orquesta los flujos de datos y la lógica de control. Se suscribe al broker MQTT para recibir los datos de potencia y los procesa según reglas predefinidas. Además de detectar anomalías, este servicio gestiona la \textbf{automatización temporal}: mediante nodos de inyección cronológica (timers), envía comandos de encendido o apagado a los actuadores en horarios específicos (ej. corte de energía en horario laboral o nocturno) para optimizar el consumo.
            \item \textbf{InfluxDB:} Es la base de datos optimizada para series temporales.
            Almacena eficientemente los datos de consumo desagregados por nodo (Habitación 1, Habitación 2) con su marca de tiempo para consultas históricas.
            \item \textbf{Grafana:} Es la plataforma de visualización. Se conecta a InfluxDB como fuente de datos y permite crear un dashboard web con gráficas, medidores e indicadores del consumo en tiempo real e histórico por zona.
        \end{itemize}

        \paragraph{3. Subsistema de Control y Notificación:}
        Son los componentes que interactúan con el usuario o con la instalación eléctrica de forma activa.
        \begin{itemize}
            \item \textbf{Comunicación por Telegram:} La integración se realiza directamente desde el servidor local utilizando la comunicación mediante Telegram. La Raspberry Pi procesa los datos y envía alertas puntuales o recibe comandos de usuario, manteniendo la privacidad de los datos históricos.
            
            \item \textbf{Actuadores Inteligentes (Smart Switches):}
            En sustitución de un esquema centralizado, se implementan módulos de interrupción WiFi basados en microcontroladores (tipo Sonoff Basic) instalados en cada habitación. Estos dispositivos actúan como relevadores inalámbricos conectados al servidor. Su función es ejecutar las órdenes de corte o reconexión provenientes de Node-RED, ya sean disparadas por una alerta de seguridad, una decisión manual del usuario o una \textbf{rutina de horario programado}, permitiendo una gestión eficiente y granular de la energía por zonas.
        \end{itemize}

    \subsection{Especificaciones Técnicas}
        El prototipo a desarrollar buscará cumplir con las siguientes especificaciones:
        \begin{itemize}
            \item \textbf{Rango de Medición:} 0 – 100 Amperes (AC) por nodo.
            \item \textbf{Voltaje de Operación (Medición):} 127V AC (Monofásico).
            \item \textbf{Precisión Esperada (Potencia):} $\pm 5\%$ comparado con un medidor comercial.
            \item \textbf{Frecuencia de Muestreo (ESP32):} Envío de datos al servidor cada 10 segundos.
            \item \textbf{Plataforma (Servidor):} Raspberry Pi 3B+ o superior.
            \item \textbf{Protocolo de Comunicación:} MQTT sobre WiFi (red local).
            \item \textbf{Interfaz de Usuario:} Dashboard en Grafana accesible vía navegador web en la red local.
            \item \textbf{Sistema de Control:} Actuación distribuida mediante interruptores WiFi (Smart Switches) de 10A, controlados vía Wi-Fi.
            \item \textbf{Automatización:} Capacidad de programación de horarios de encendido y apagado automático por nodos (Habitación).
        \end{itemize}
\newpage
\section{Desarrollo del proyecto}
\newpage

\section{Resultados}
\newpage

\section{Análisis de resultados}
\newpage

\section{Conclusiones}

 % Comentarios de código para posibles figuras o tablas futuras
\begin{comment}
 %imagenes
    \begin{figure}[H]
        \centering
        \includegraphics[width=0.8\textwidth]{img/PPG.png}
        \caption[Diagrama de la fotopletismografía]{Diagrama de la fotopletismografía\footnotemark}
        \label{fig:PPG}
    \end{figure}
    \footnotetext{Diagrama de la fotopletismografía. Imagen tomada de Blog da robotica. Fuente: \url{https://www.blogdarobotica.com/}}
% tablas
    \begin{table}[t]
        \begin{center}
        \begin{tabular}{ l | c | c}
            Aproximación & Factor de calidad Q & Constante K\\ \hline
            Butterworth & 0.7071 & 1.0000\\
            Chebyshev (cresta de 0.01db) & 0.7247 & 1.0231\\
            Chebyshev (cresta de 0.1db) & 0.7673 & 1.0674 \\
            Chebyshev (cresta de 0.25db) &  0.8093 & 1.0991 \\
            Chebyshev (cresta de 0.5db) & 0.8638 & 1.1286 \\
            Chebyshev (cresta de 1db) & 0.9564 & 1.1596 \\
            Bessel & 0.5771 & 0.7840 \\
            \end{tabular}
        \caption{Constantes de los filtros}
        \label{tab:constantes}
        \end{center}
    \end{table} 
\end{comment}