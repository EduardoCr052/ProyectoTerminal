\setcounter{page}{1} % Iniciar la numeración de las páginas en este punto.
\pagestyle{fancy}

\section{Resumen}

El consumo eléctrico residencial suele gestionarse de manera reactiva debido a la falta de herramientas que permitan desglosar y analizar el gasto energético en tiempo real. Este Proyecto presenta el diseño e implementación de un sistema de gestión inteligente basado en el Internet de las Cosas (IoT), cuyo propósito es dotar al usuario de capacidades de monitoreo granular y control activo sobre su instalación.

La solución técnica consiste en una arquitectura distribuida de nodos de adquisición independientes, basados en microcontroladores ESP32, que permiten la medición desagregada de corriente por habitaciones o circuitos específicos para la detección de consumo fantasma. La comunicación del sistema se fundamenta en el protocolo MQTT, centralizando el flujo de información en un servidor local desplegado en una Raspberry Pi, donde se integran servicios de lógica de control (Node-RED), bases de datos de series temporales (InfluxDB) y visualización web (Grafana). Adicionalmente, el sistema cierra el lazo de gestión mediante la integración de actuadores WiFi (Smart Switches) y algoritmos de automatización, permitiendo la ejecución de horarios programados y el corte remoto del suministro, validando así una alternativa accesible y privada para la eficiencia energética en el hogar.
\newpage

\section{Introducción}
    En el contexto global actual, la eficiencia energética se ha consolidado como un pilar fundamental para el desarrollo sostenible, impulsada tanto por la necesidad de mitigar el impacto ambiental como por la optimización de los recursos económicos \cite{chevez2018energias}. El sector residencial representa una porción significativa del consumo total de energía eléctrica, sin embargo, la mayoría de los usuarios carecen de herramientas que les permitan comprender y gestionar su consumo de manera efectiva. La facturación eléctrica tradicional ofrece únicamente un resumen mensual, funcionando como un indicador tardío que no permite identificar patrones de uso específicos ni detectar fuentes de desperdicio, como el conocido consumo fantasma o en modo de espera (standby) \cite{iea2022standby}. La problemática central que se aborda es la falta de acceso a información granular y en tiempo real sobre el consumo de energía en el hogar, así como la ausencia de mecanismos accesibles para actuar sobre dicho consumo en caso de anomalías o necesidades de gestión remota. Esta carencia impide a los usuarios tomar decisiones informadas y acciones directas para optimizar el uso de sus aparatos y reducir su gasto energético. Para dar solución a esta necesidad, se propone el diseño y la implementación de un \textbf{sistema inteligente de monitoreo y control energético residencial}, basado en hardware de bajo costo y software de código abierto. El desarrollo del proyecto se fundamentará en la integración de varias tecnologías clave. En primer lugar, el \textbf{sensado de corriente no invasivo} \cite{fraden2016handbook}, que utiliza un transformador de corriente (SCT-013) para medir el flujo eléctrico de forma segura. En segundo lugar, se implementará una arquitectura de \textbf{medición distribuida} mediante una red de \textbf{nodos embebidos} (ESP32) \cite{vahid2002embedded}. Cada nodo actuará como punto de adquisición independiente en distintas zonas o habitaciones, procesando la señal de su respectivo sensor y transmitiendo los datos simultáneamente a través del protocolo de mensajería \textbf{MQTT}, estándar en el Internet de las Cosas (IoT) por su eficiencia. Finalmente, se configurará un \textbf{servidor local en una Raspberry Pi} que centralizará la gestión del sistema: el broker \textbf{EMQX} recibirá los datos, \textbf{Node-RED} orquestará el flujo de información para su procesamiento, detección de anomalías y gestión de horarios automatizados, enviando \textbf{notificaciones vía Telegram} cuando sea necesario. Una base de datos \textbf{InfluxDB} almacenará las series temporales de consumo, y \textbf{Grafana} proporcionará una interfaz web alojada en la misma \textbf{Raspberry Pi} para la \textbf{visualización} de los datos en tiempo real e históricos. Este sistema no solo proporcionará una medición precisa y desagregada del consumo, sino que permitirá la \textbf{detección inteligente de anomalías} y ofrecerá al usuario la capacidad de \textbf{controlar remotamente el suministro eléctrico} por zonas mediante \textbf{actuadores inteligentes WiFi}, añadiendo una capa de gestión activa, automatización y seguridad al monitoreo energético del hogar. \newpage

\section{Antecedentes}

    El monitoreo del consumo de energía eléctrica mediante tecnologías del IoT ha sido abordado en diversos trabajos académicos a nivel nacional e internacional. Para contextualizar la presente propuesta y destacar su originalidad, se analizan a continuación cinco trabajos relevantes que representan el estado actual de la tecnología. \subsection{Trabajo de Grado CUC: Diseño de un prototipo de un sistema domótico} Este trabajo describe el diseño de un sistema domótico que incluye la monitorización del consumo eléctrico en una vivienda. El prototipo utiliza sensores conectados a un microcontrolador y visualiza los datos a través de una aplicación móvil desarrollada específicamente para el proyecto. \cite{AhumadaBarros2021} 

        \paragraph{Semejanzas:} Comparte el enfoque residencial y el uso de una interfaz gráfica para que el usuario visualice su consumo.
        \paragraph{Diferencias y Aportación:} El trabajo se centra en la medición y visualización mediante una app a medida.
        Nuestra propuesta se diferencia al implementar una arquitectura de servidor local (Raspberry Pi) con un stack de software industrial (Grafana/InfluxDB) y añade capacidades de control activo (corte de suministro) y notificaciones por Telegram, ausentes en este antecedente.
    \subsection{Paper IEEE ROPEC 2021: Smart IoT Device For Energy Consumption Monitoring}
    Este trabajo propone un dispositivo IoT basado en ESP8266 que transmite datos a un servidor web (PHP/MySQL).
    Su característica principal es el uso de redes neuronales para identificar qué electrodoméstico específico está consumiendo energía (técnica NILM).
    \cite{LopezAlfaro2021}

        \paragraph{Semejanzas:} Uso de microcontroladores ESP y monitoreo en tiempo real vía WiFi.
        \paragraph{Diferencias y Aportación:} Este paper se enfoca en la identificación de cargas mediante IA.
        Nuestro proyecto prioriza la gestión y seguridad: detección de anomalías de consumo general y la capacidad de actuación remota (corte de energía).
        Además, utilizamos una base de datos de series de tiempo (InfluxDB), más eficiente que la base SQL del paper.
    \subsection{Paper IEEE ROPEC 2019: Energy Monitoring Consumption at IoT-Edge}
    Este trabajo propone un circuito de monitoreo con sensor SCT-013 y ESP8266.
    Implementa una red neuronal en el propio microcontrolador para detectar consumos inusuales, enviando los datos a una base de datos MySQL en una Raspberry Pi.
    \cite{AguirreNunez2019}

        \paragraph{Semejanzas:} Coincide plenamente en el hardware (SCT-013, ESP, RPi) y el objetivo de detectar anomalías.
        \paragraph{Diferencias y Aportación:} La diferencia es el método de detección y almacenamiento. El paper usa IA embebida y MySQL.
        Nosotros delegamos la lógica a Node-RED en el servidor (facilitando la configuración de reglas) y usamos InfluxDB.
        Aportamos la capacidad crítica de control remoto que este trabajo no contempla.
    \subsection{Trabajo Final UNLP: Diseño e Implementación de un Sistema de Medición...}
    Este proyecto detalla un sistema de medición residencial utilizando hardware de código abierto.
    Los datos son procesados para su presentación al usuario mediante una interfaz web local.
    \cite{Cicutti2019}

        \paragraph{Semejanzas:} Uso de hardware accesible y filosofía de bajo costo.
        \paragraph{Diferencias y Aportación:} Nuestra propuesta expande este concepto integrando la orquestación de flujos de datos con Node-RED y un sistema de alertas proactivas.
        A diferencia de la visualización pasiva de este trabajo, nuestro sistema permite al usuario intervenir físicamente en la instalación a través de actuadores distribuidos.
    \subsection{Proyecto de Código Abierto: OpenEnergyMonitor}
    Plataforma de referencia mundial en hardware y software abierto.
    Permite monitorear consumo, generación y controlar cargas mediante su software EmonCMS.
    \cite{openenergymonitor}

        \paragraph{Semejanzas:} Base de inspiración para el uso de sensores no invasivos y la filosofía de código abierto.
        \paragraph{Diferencias y Aportación:} OpenEnergyMonitor es un ecosistema complejo y modular.
        Nuestra aportación es la integración de una solución 'todo en uno' más accesible, utilizando herramientas genéricas y populares (Telegram, Grafana) en lugar de un CMS especializado.
    \newpage 

    \subsection{Comparación Cualitativa de Trabajos Relacionados}
    
     La Tabla \ref{tab:comparacion_antecedentes} resume las similitudes y diferencias clave entre los trabajos analizados y la propuesta presentada, destacando las contribuciones originales de este proyecto en el contexto del monitoreo y control energético residencial.
    \begin{table}[H] 
        \centering
        \caption{Comparación cualitativa de los trabajos relacionados.}
        \label{tab:comparacion_antecedentes}
        \small 
        \resizebox{\textwidth}{!}{%
        \begin{tabular}{>{\centering\arraybackslash}p{2cm} p{5.5cm} p{5.5cm}} 
        \toprule
        \textbf{Ref.} & \textbf{Similitudes} & \textbf{Diferencias} \\
        \midrule

        \cite{AhumadaBarros2021} & 
   
        \begin{itemize}[leftmargin=*, topsep=0pt, itemsep=0pt, parsep=0pt, partopsep=0pt]
            \item Enfoque residencial.
            \item Interfaz en App móvil.
        \end{itemize} & 
        \begin{itemize}[leftmargin=*, topsep=0pt, itemsep=0pt, parsep=0pt, partopsep=0pt]
            \item Software a medida vs. Stack estándar.
            \item Sin control activo.
        \end{itemize} \\
        \midrule

        \cite{LopezAlfaro2021} & 
        \begin{itemize}[leftmargin=*, topsep=0pt, itemsep=0pt, parsep=0pt, partopsep=0pt]
            \item Uso de ESP y WiFi.
            \item Monitoreo tiempo real.
        \end{itemize} & 
        \begin{itemize}[leftmargin=*, topsep=0pt, itemsep=0pt, parsep=0pt, partopsep=0pt]
            \item Enfoque en IA (Identificación) vs. Control/Seguridad.
            \item SQL vs. InfluxDB.
        \end{itemize} \\
        \midrule

        \cite{AguirreNunez2019} & 
        \begin{itemize}[leftmargin=*, topsep=0pt, itemsep=0pt, parsep=0pt, partopsep=0pt]
            \item Hardware idéntico (SCT, ESP, RPi).
            \item Detección anomalías.
        \end{itemize} & 
        \begin{itemize}[leftmargin=*, topsep=0pt, itemsep=0pt, parsep=0pt, partopsep=0pt]
            \item IA embebida vs. Lógica Node-RED.
            \item Solo monitoreo vs. Monitoreo + Control.
        \end{itemize} \\
        \midrule

        \cite{Cicutti2019} & 
        \begin{itemize}[leftmargin=*, topsep=0pt, itemsep=0pt, parsep=0pt, partopsep=0pt]
            \item Hardware abierto.
            \item Visualización web local.
        \end{itemize} & 
        \begin{itemize}[leftmargin=*, topsep=0pt, itemsep=0pt, parsep=0pt, partopsep=0pt]
            \item Visualización pasiva vs. Gestión activa.
            \item Sin notificaciones.
        \end{itemize} \\
        \midrule

        \cite{openenergymonitor} & 
        \begin{itemize}[leftmargin=*, topsep=0pt, itemsep=0pt, parsep=0pt, partopsep=0pt]
            \item Sensado no invasivo.
            \item Plataforma auto-alojable.
        \end{itemize} & 
        \begin{itemize}[leftmargin=*, topsep=0pt, itemsep=0pt, parsep=0pt, partopsep=0pt]
            \item Complejo/modular vs. Integrado.
            \item EmonCMS vs. Stack MQTT/Node-RED.
        \end{itemize} \\

        \bottomrule
        \end{tabular}
        }
    \end{table}
\newpage
    
\section{Justificación}

    El análisis de las opciones actuales para el monitoreo energético residencial revela una limitación significativa: la mayoría de las herramientas se restringen a una visualización pasiva de datos o presentan barreras de alto costo y complejidad técnica.
    Esta carencia deja al usuario promedio sin capacidad de respuesta ante anomalías o desperdicios en su consumo.
    La propuesta justifica su realización al cubrir este vacío mediante un sistema que trasciende la simple observación para integrar el control activo.
    A diferencia de los medidores convencionales, este proyecto cierra el lazo de gestión permitiendo al usuario no solo ver su consumo, sino actuar sobre él mediante el corte remoto del suministro y la recepción de alertas automáticas.
    La implementación de una arquitectura basada en un servidor local es clave para resolver los problemas de privacidad y dependencia de servicios externos, garantizando que los datos sensibles permanezcan bajo el control del usuario.
    Desde la perspectiva de la ingeniería, el proyecto demuestra la viabilidad de democratizar tecnologías avanzadas de gestión energética utilizando hardware accesible y software de código abierto.
    Esta solución equilibra la robustez técnica con la economía, entregando a los hogares una herramienta efectiva de seguridad y eficiencia que supera las limitaciones funcionales de los sistemas puramente informativos.
\newpage

\section{Objetivos}

    \subsection{Objetivo General}
        Diseñar e Implementar un sistema electrónico inteligente para el monitoreo y control del consumo energético residencial, para proporcionar al usuario una herramienta accesible de gestión remota del suministro eléctrico.
    \subsection{Objetivos Particulares}
    \begin{enumerate}
        \item \textbf{Diseñar} el circuito electrónico de acondicionamiento para la señal del sensor de corriente no invasivo SCT-013, asegurando que la salida de voltaje se mantenga dentro del rango operativo de 0V a 3.3V del convertidor analógico-digital (ADC) del ESP32.
        \item \textbf{Construir} los nodos de medición distribuidos, ensamblando para cada uno su respectivo sensor SCT-013, el circuito de acondicionamiento y el microcontrolador ESP32, permitiendo así la monitorización independiente de dos zonas (habitaciones) en el prototipo.
        \item \textbf{Diseñar e implementar} el firmware para el ESP32 que adquiera las lecturas del sensor, calcule la potencia eléctrica instantánea y publique los resultados vía MQTT al servidor local.
        \item \textbf{Establecer} el servidor local en la Raspberry Pi, instalando y configurando el broker MQTT, la base de datos InfluxDB y el sistema de visualización Grafana.
        \item \textbf{Programar} en Node-RED los flujos de trabajo para procesar los datos MQTT, almacenarlos en InfluxDB, detectar condiciones de consumo anómalo y ejecutar rutinas de automatización, incluyendo la programación de horarios de corte y reconexión del suministro.
        \item \textbf{Implementar} el sistema de notificación de alertas por consumo anómalo mediante una comunicación con Telegram controlado desde Node-RED.
        \item \textbf{Integrar} actuadores inteligentes WiFi al sistema, configurando su comunicación MQTT con el servidor local para permitir el corte de suministro remoto, ya sea por alertas de seguridad, comandos manuales o itinerarios programados automáticamente.
        \item \textbf{Verificar} que la precisión del sistema de medición se encuentre dentro de un margen de error del $\pm 5\%$ en comparación con un wattmetro comercial, bajo diferentes perfiles de carga residencial.
        \item \textbf{Validar} la funcionalidad completa del sistema, verificando la correcta operación de la visualización en Grafana, las notificaciones en Telegram y la activación inalámbrica de los actuadores.
    \end{enumerate}
\newpage

\section{Marco Teórico}

    El desarrollo de sistemas de monitoreo energético requiere una comprensión profunda de la interacción entre la teoría de circuitos de corriente alterna, la instrumentación electrónica y las arquitecturas de comunicación en red. En este apartado se establecen los fundamentos teóricos necesarios para el diseño del sistema, abarcando desde los principios matemáticos de la medición de potencia hasta los protocolos de comunicación estándar en el Internet de las Cosas.

    \subsection{Fundamentos de Medición Eléctrica en Corriente Alterna}
        Para gestionar eficientemente la energía eléctrica, es imperativo comprender la naturaleza dinámica de las variables físicas involucradas en un sistema residencial monofásico, donde la tensión y la corriente varían constantemente en el tiempo.

        \subsubsection{Potencia Instantánea}
        En un circuito de corriente alterna (AC), la potencia no es un valor estático, sino una función del tiempo. Se define la potencia instantánea $p(t)$ como el producto del voltaje instantáneo $v(t)$ y la corriente instantánea $i(t)$ en un momento dado:

        \begin{equation}
            p(t) = v(t) \cdot i(t)
        \end{equation}

        Considerando que en la red eléctrica de México el voltaje se comporta como una señal sinusoidal $v(t) = V_{pico} \sin(\omega t)$ y asumiendo una carga lineal donde la corriente tiene un desfase $\phi$, la ecuación se expande a:

        \begin{equation}
            p(t) = V_{pico} I_{pico} \sin(\omega t) \sin(\omega t - \phi)
        \end{equation}

        Esta variable es crítica para el algoritmo del microcontrolador, ya que el ESP32 debe realizar la multiplicación de estas muestras discretas miles de veces por segundo para obtener una representación fiel del consumo real antes de promediar.

        \subsubsection{Valor Eficaz (RMS)}
        Dado que el promedio de una onda sinusoidal pura es cero, la potencia instantánea por sí sola es difícil de interpretar para el usuario final. Para cuantificar la capacidad efectiva de la corriente para realizar trabajo (disipar calor), se utiliza el valor de la Raíz Cuadrática Media (RMS). 
        
        El sistema propuesto implementa el algoritmo digital para calcular el voltaje y corriente RMS a partir de $N$ muestras tomadas durante un ciclo de red (16.6 ms a 60Hz):

        \begin{equation}
            I_{RMS} = \sqrt{\frac{1}{N} \sum_{n=0}^{N-1} i^2[n]}
        \end{equation}

        Este valor RMS es el estándar utilizado por la Comisión Federal de Electricidad (CFE) y los multímetros comerciales, permitiendo validar la precisión del prototipo.

        \subsubsection{Potencia Aparente ($S$)}
        La Potencia Aparente es la magnitud total de potencia que debe suministrar la red eléctrica a la vivienda, independientemente de si se consume útilmente o se regresa a la red (reactiva). Se representa con la letra $S$ y su unidad es el Volt-Ampere (VA).

        Matemáticamente, se obtiene del producto de los valores eficaces calculados previamente:

        \begin{equation}
            S = V_{RMS} \cdot I_{RMS}
        \end{equation}

        En el contexto de este proyecto, la medición de Potencia Aparente es la métrica principal para el monitoreo, ya que representa la "carga" total que los electrodomésticos imponen al sistema, siendo ideal para detectar consumos anómalos o dispositivos encendidos innecesariamente (consumo fantasma).

    \subsection{Instrumentación y Sensores (SCT-013)}
        La medición de corriente se realiza mediante el método no invasivo utilizando sensores de la serie SCT-013, los cuales operan bajo el principio de los Transformadores de Corriente (CT). En esta configuración, el conductor de la instalación eléctrica actúa como el **devanado primario** (una sola espira), mientras que el sensor integra un núcleo de ferrita partido y un **devanado secundario** con un número elevado de vueltas (típicamente 2000), lo que establece la relación de transformación de corriente \cite{naylamp2023sct}.

        \textbf{Principio de Transducción:}
        La corriente que circula por el primario induce un flujo magnético en el núcleo, el cual genera una corriente proporcionalmente menor en el secundario. Existen dos variantes principales en esta familia de sensores:
        \begin{itemize}
            \item \textbf{Salida de Corriente (SCT-013-000):} Entrega una señal de corriente (ej. 50 mA por cada 100 A). Requiere una resistencia de carga externa (\textit{Burden Resistor}) para convertir esta corriente en una señal de voltaje legible por el microcontrolador.
            \item \textbf{Salida de Voltaje (SCT-013-030):} Incluye una resistencia de carga interna, entregando directamente una señal de voltaje (ej. 1 V por cada 30 A) \cite{naylamp2023sct}.
        \end{itemize}

        \textbf{Acondicionamiento de Señal (Offset DC):}
        Dado que la salida del sensor es una señal de corriente alterna (AC) que oscila entre valores positivos y negativos, y considerando que el convertidor analógico-digital (ADC) del ESP32 opera únicamente con voltajes positivos (0V a 3.3V), es necesario implementar un circuito de acondicionamiento.
        
        Este circuito debe añadir un voltaje de desplazamiento (\textit{DC Offset}) a la señal, típicamente de $V_{CC}/2$ (1.65V para el ESP32), centrando la onda sinusoidal en el rango dinámico del ADC. Sin este acondicionamiento, los semiciclos negativos de la corriente alterna serían recortados o podrían dañar la entrada del microcontrolador \cite{naylamp2023sct}.

        Para el modelo SCT-013-000 seleccionado (100A), la resistencia \textit{burden} ($R_{carga}$) necesaria para obtener un voltaje pico deseado ($V_{pico}$) se calcula mediante la Ley de Ohm aplicada al secundario:
        \begin{equation}
            R_{carga} = \frac{V_{pico}}{I_{secundario}}
        \end{equation}
        Donde $I_{secundario}$ es la corriente de salida nominal del sensor (0.05 A).
        

    \subsection{Hardware del Sistema}
        \textbf{Microcontrolador ESP32:} Se selecciona este SoC (System on Chip) por su capacidad de procesamiento dual-core y conectividad WiFi integrada. Su convertidor analógico-digital (ADC) de 12 bits permite una resolución suficiente para digitalizar la señal proveniente de los sensores de corriente.
        
        \textbf{Raspberry Pi:} Actúa como el servidor de borde (\textit{Edge Server}). Es una computadora de placa reducida basada en Linux capaz de ejecutar el stack de software del servidor (Broker, Base de Datos y Dashboard) las 24 horas con un consumo energético mínimo.
        
        \textbf{Actuadores Inteligentes (Smart Switches):} Dispositivos basados en microcontroladores (tipo Sonoff Basic R4) que integran un relevador electromecánico y conectividad WiFi. Su función es interrumpir físicamente el circuito eléctrico al recibir un comando digital, permitiendo el control ON/OFF remoto.

        \subsection{Plataformas de Software}
        \textbf{Node-RED:} Herramienta de programación visual basada en flujos. Actúa como el orquestador lógico del sistema, procesando los mensajes MQTT, gestionando las reglas de automatización (horarios) y enrutando los datos.
        
        \textbf{InfluxDB:} Base de datos de series temporales (TSDB) optimizada para almacenar datos con marcas de tiempo precisas, ideal para registrar el historial de consumo eléctrico.
        
        \textbf{Grafana:} Plataforma de visualización que permite crear tableros de control (\textit{dashboards}) interactivos para que el usuario final monitoree las variables eléctricas y el estado de los actuadores en tiempo real.
\newpage

\section{Desarrollo del proyecto}
\newpage

\section{Resultados}
\newpage

\section{Análisis de resultados}
\newpage

\section{Conclusiones}

 % Comentarios de código para posibles figuras o tablas futuras
\begin{comment}
 %imagenes
    \begin{figure}[H]
        \centering
        \includegraphics[width=0.8\textwidth]{img/PPG.png}
        \caption[Diagrama de la fotopletismografía]{Diagrama de la fotopletismografía\footnotemark}
        \label{fig:PPG}
    \end{figure}
    \footnotetext{Diagrama de la fotopletismografía. Imagen tomada de Blog da robotica. Fuente: \url{https://www.blogdarobotica.com/}}
% tablas
    \begin{table}[t]
        \begin{center}
        \begin{tabular}{ l | c | c}
            Aproximación & Factor de calidad Q & Constante K\\ \hline
            Butterworth & 0.7071 & 1.0000\\
            Chebyshev (cresta de 0.01db) & 0.7247 & 1.0231\\
            Chebyshev (cresta de 0.1db) & 0.7673 & 1.0674 \\
            Chebyshev (cresta de 0.25db) &  0.8093 & 1.0991 \\
            Chebyshev (cresta de 0.5db) & 0.8638 & 1.1286 \\
            Chebyshev (cresta de 1db) & 0.9564 & 1.1596 \\
            Bessel & 0.5771 & 0.7840 \\
            \end{tabular}
        \caption{Constantes de los filtros}
        \label{tab:constantes}
        \end{center}
    \end{table} 
\end{comment}