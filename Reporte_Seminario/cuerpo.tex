\setcounter{page}{1} % Iniciar la numeración de las páginas en este punto.
\pagestyle{fancy}

\section{Introducción}
    En el contexto global actual, la eficiencia energética se ha consolidado como un pilar fundamental para el desarrollo sostenible, impulsada tanto por la necesidad de mitigar el impacto ambiental como por la optimización de los recursos económicos \cite{chevez2018energias}. El sector residencial representa una porción significativa del consumo total de energía eléctrica, sin embargo, la mayoría de los usuarios carecen de herramientas que les permitan comprender y gestionar su consumo de manera efectiva. La facturación eléctrica tradicional ofrece únicamente un resumen mensual, funcionando como un indicador tardío que no permite identificar patrones de uso específicos ni detectar fuentes de desperdicio, como el conocido consumo fantasma o en modo de espera (standby) \cite{iea2022standby}.

    La problemática central que se aborda es la falta de acceso a información granular y en tiempo real sobre el consumo de energía en el hogar, así como la ausencia de mecanismos accesibles para actuar sobre dicho consumo en caso de anomalías o necesidades de gestión remota. Esta carencia impide a los usuarios tomar decisiones informadas y acciones directas para optimizar el uso de sus aparatos y reducir su gasto energético. Para dar solución a esta necesidad, se propone el diseño e implementación de un sistema inteligente de monitoreo y control energético residencial, basado en hardware de bajo costo y software de código abierto.

    El desarrollo del proyecto se fundamentará en la integración de varias tecnologías clave. En primer lugar, el sensado de corriente no invasivo \cite{fraden2016handbook}, que utiliza un transformador de corriente (SCT-013) para medir el flujo eléctrico de forma segura. En segundo lugar, un sistema embebido (ESP32) \cite{vahid2002embedded} actuará como nodo de adquisición, procesando la señal del sensor y transmitiendo los datos a través del protocolo de mensajería MQTT, estándar en el Internet de las Cosas (IoT) por su eficiencia. Finalmente, se configurará un servidor local en una Raspberry Pi que centralizará la gestión del sistema: un broker MQTT recibirá los datos, Node-RED orquestará el flujo de información para su procesamiento, almacenamiento, detección de anomalías y envío de notificaciones vía Telegram, y Grafana proporcionará una interfaz web para la visualización de los datos en tiempo real e históricos.

    Este sistema no solo proporcionará una medición precisa del consumo, sino que, a través del análisis de patrones en Node-RED, permitirá la detección inteligente de anomalías y ofrecerá al usuario la capacidad de controlar remotamente el suministro eléctrico principal mediante un contactor, añadiendo una capa de gestión activa y seguridad al monitoreo energético del hogar.
\newpage % Nueva página

\section{Antecedentes}

    El monitoreo del consumo de energía eléctrica mediante tecnologías del Internet de las Cosas (IoT) ha sido abordado en diversos trabajos académicos a nivel nacional e internacional. Para contextualizar la presente propuesta y destacar su originalidad, se analizan a continuación tres trabajos relevantes.

    \subsection{Trabajo de Grado CUC: "Diseño de un prototipo de un sistema domótico para la monitorización del consumo de energía eléctrica..." \cite{AhumadaBarros2021}}
    Este trabajo, realizado en la Universidad de la Costa (CUC), Colombia, describe el diseño de un sistema domótico que incluye la monitorización del consumo eléctrico en una vivienda. El prototipo utiliza sensores conectados a un microcontrolador y visualiza los datos a través de una aplicación móvil desarrollada para tal fin, permitiendo al usuario observar el consumo de diferentes electrodomésticos \cite{AhumadaBarros2021}.

    \paragraph{Semejanzas:} Comparte con nuestra propuesta el enfoque en el sector residencial y el objetivo de proporcionar al usuario información sobre su consumo eléctrico a través de una interfaz accesible. Ambos proyectos reconocen la importancia de la monitorización para la gestión energética en el hogar.

    \paragraph{Diferencias y Aportación del Proyecto:} Las diferencias clave residen en la arquitectura del sistema y el alcance funcional. Mientras que el trabajo de Ahumada y Barros se centra en la visualización a través de una app móvil, posiblemente conectada a una nube o a un sistema domótico específico, nuestra propuesta implementa una **arquitectura de procesamiento local robusta basada en una Raspberry Pi**. La aportación distintiva de nuestro proyecto es la **integración de herramientas de código abierto como Node-RED y Grafana en un servidor local**, lo que permite no solo la visualización sino también el **procesamiento avanzado de datos, la detección de anomalías y el envío de notificaciones proactivas a través de Telegram**. Adicionalmente, nuestro sistema incorpora una **capacidad de control activo**, permitiendo la interrupción remota del suministro eléctrico mediante un contactor, funcionalidad orientada a la seguridad y gestión que no se detalla en el antecedente.

    \subsection{Trabajo Final UNLP: "Diseño e Implementación de un Sistema de Medición y Visualización del Consumo Eléctrico Residencial" \cite{Cicutti2019}}
    Este trabajo final de la Universidad Nacional de La Plata (UNLP), Argentina, detalla el desarrollo de un sistema para medir y visualizar el consumo eléctrico residencial utilizando hardware de código abierto y sensores de corriente. Los datos son procesados y presentados al usuario a través de una interfaz web local o una plataforma de visualización \cite{Cicutti2019}.

    \paragraph{Semejanzas:} Al igual que nuestra propuesta, este proyecto emplea hardware accesible (posiblemente microcontroladores y sensores similares) y busca ofrecer una herramienta de visualización del consumo para usuarios residenciales como alternativa económica.

    \paragraph{Diferencias y Aportación del Proyecto:} Aunque ambos proyectos pueden utilizar interfaces web locales, nuestra propuesta se distingue por la **orquestación explícita del flujo de datos mediante Node-RED y la visualización especializada con Grafana**, herramientas estándar en entornos industriales y de IoT para el manejo y presentación de series temporales. La aportación central de nuestro trabajo no es solo la medición, sino la **creación de un sistema inteligente local capaz de analizar patrones, generar alertas contextualizadas vía Telegram y ejecutar acciones de control sobre el suministro eléctrico**, conformando un ecosistema de gestión energética más completo y proactivo que una simple interfaz de visualización.

    \subsection{Proyecto de Código Abierto: "OpenEnergyMonitor" \cite{openenergymonitor}}
    OpenEnergyMonitor es un proyecto internacional de hardware y software de código abierto para el monitoreo de energía. Su plataforma permite medir consumo, generación y controlar cargas, utilizando hardware modular y su software EmonCMS, una potente plataforma web auto-alojable \cite{openenergymonitor}.

    \paragraph{Semejanzas:} Comparte la filosofía de código abierto, el uso de sensores no invasivos y el objetivo de proporcionar herramientas para entender el uso de la energía. La capacidad de auto-alojamiento de EmonCMS es conceptualmente similar a nuestro enfoque de servidor local.

    \paragraph{Diferencias y Aportación del Proyecto:} OpenEnergyMonitor, con EmonCMS, ofrece una solución muy completa pero también compleja, orientada a usuarios con conocimientos técnicos avanzados. La aportación de nuestra propuesta radica en utilizar una **combinación diferente y muy popular de herramientas de código abierto (MQTT, Node-RED, Grafana)** que, si bien puede no tener todas las funcionalidades de EmonCMS, ofrece una **curva de aprendizaje más accesible y una gran flexibilidad para la integración con otros sistemas domóticos o de notificación (como Telegram)**. Nuestro enfoque se centra en construir un sistema de monitoreo *y control* inteligente, funcional y fácil de replicar, utilizando un stack de software ampliamente adoptado en la comunidad IoT actual.
    \newpage % Nueva página
    
\section{Justificación}
    La gestión eficiente de la energía eléctrica en el sector residencial constituye un desafío significativo, originado principalmente por la falta de acceso a información detallada y en tiempo real sobre el consumo, así como por la ausencia de mecanismos accesibles para actuar remotamente sobre el suministro. El problema central afecta a los usuarios domésticos, quienes reciben una facturación mensual insuficiente para comprender sus patrones de uso, identificar aparatos ineficientes o cuantificar el "consumo fantasma" \cite{sap2022smartmetering}. Esta carencia de datos y control representa una barrera para implementar estrategias efectivas de ahorro y gestión energética.

    Resolver esta problemática mediante la ingeniería electrónica aporta beneficios directos. La importancia del proyecto radica en empoderar al usuario con datos accionables y herramientas de control, impactando directamente en la reducción del costo del servicio eléctrico y fomentando una mayor conciencia sobre el uso de recursos energéticos. Para la Ingeniería Electrónica, este proyecto integra disciplinas clave: diseño de sistemas embebidos (ESP32), acondicionamiento de señales, comunicaciones inalámbricas (WiFi, MQTT), administración de sistemas en Linux (Raspberry Pi), programación visual (Node-RED), visualización de datos (Grafana) y control de potencia (contactor) \cite{scielo2024iot}. La aportación práctica es un prototipo funcional, seguro y de bajo costo de un sistema de gestión energética inteligente, utilizando software de código abierto y hardware accesible.

    La viabilidad del proyecto está garantizada por el uso de componentes económicos como el ESP32 y el sensor SCT-013 \cite{programarfacil2017sct013}, junto con una Raspberry Pi y herramientas de software gratuitas. La implementación es consistente con las competencias de la Licenciatura en Ingeniería Electrónica, y el tiempo de desarrollo se ajusta al cronograma del Proyecto de Integración. Los usuarios residenciales son los beneficiarios directos, obteniendo control sobre su consumo y seguridad. El prototipo puede servir como material didáctico en laboratorios para prácticas de IoT, sistemas embebidos y automatización.
    \newpage

\section{Objetivos}

    \subsection{Objetivo General}
    Desarrollar un sistema electrónico inteligente para el monitoreo y control del consumo energético residencial, mediante el uso de un sensor no invasivo, un microcontrolador ESP32, una Raspberry Pi como servidor local (con MQTT, Node-RED, Grafana) y un contactor, para proporcionar al usuario una herramienta accesible de visualización, análisis, notificación de anomalías y control remoto del suministro eléctrico.

    \subsection{Objetivos Particulares}
    Para alcanzar el objetivo general, se llevarán a cabo las siguientes etapas secuenciales:
    \begin{enumerate}
        \item \textbf{Diseñar} el circuito de acondicionamiento de señal para el sensor de corriente no invasivo SCT-013, asegurando la compatibilidad con la entrada analógica del microcontrolador ESP32.
        \item \textbf{Ensamblar} el prototipo físico del nodo de medición, integrando el sensor SCT-013, el circuito de acondicionamiento y el microcontrolador ESP32.
        \item \textbf{Programar} el firmware del ESP32 para adquirir las mediciones de corriente, calcular la potencia instantánea y transmitir los datos al broker MQTT del servidor local.
        \item \textbf{Configurar} el servidor local en la Raspberry Pi, instalando el broker MQTT, Node-RED, la base de datos InfluxDB y Grafana para la recepción, procesamiento y visualización de los datos.
        \item \textbf{Implementar} en Node-RED los flujos para el análisis de datos, la detección de patrones de consumo anómalo, el envío de notificaciones vía Telegram y la lógica de control del contactor eléctrico.
        \item \textbf{Validar} la precisión de las mediciones del sistema comparándolas con un wattmetro comercial bajo distintas cargas eléctricas residenciales.
        \item \textbf{Verificar} la correcta operación del sistema de notificaciones por Telegram y la funcionalidad del control remoto del contactor eléctrico.
    \end{enumerate}
\newpage

%\section{Marco Teórico}
%\section{Metodología}
%\section{Desarrollo del proyecto}
    

% Comentarios de código para posibles figuras o tablas futuras

\begin{comment}
    \begin{figure}[H]
        \centering
        \includegraphics[width=0.8\textwidth]{img/PPG.png}
        \caption[Diagrama de la fotopletismografía]{Diagrama de la fotopletismografía\footnotemark}
        \label{fig:PPG}
    \end{figure}
    \footnotetext{Diagrama de la fotopletismografía. Imagen tomada de Blog da robotica. Fuente: \url{https://www.blogdarobotica.com/}}
\end{comment}

\begin{comment}
    \begin{table}[t]
        \begin{center}
        \begin{tabular}{ l | c | c}
            Aproximación & Factor de calidad Q & Constante K\\ \hline
            Butterworth & 0.7071 & 1.0000\\
            Chebyshev (cresta de 0.01db) & 0.7247 & 1.0231\\
            Chebyshev (cresta de 0.1db) & 0.7673 & 1.0674 \\
            Chebyshev (cresta de 0.25db) &  0.8093 & 1.0991 \\
            Chebyshev (cresta de 0.5db) & 0.8638 & 1.1286 \\
            Chebyshev (cresta de 1db) & 0.9564 & 1.1596 \\
            Bessel & 0.5771 & 0.7840 \\
            \end{tabular}
        \caption{Constantes de los filtros}
        \label{tab:constantes}
        \end{center}
    \end{table} 
\end{comment}