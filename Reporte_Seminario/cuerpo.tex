\setcounter{page}{1} % Iniciar la numeración de las páginas en este punto.
\pagestyle{fancy}

% --- SECCIÓN 1: INTRODUCCIÓN ---
\section{Introducción}

En el contexto global actual, la eficiencia energética se ha consolidado como un pilar fundamental para el desarrollo sostenible, impulsada tanto por la necesidad de mitigar el impacto ambiental como por la optimización de los recursos económicos \cite{chevez2018energias}. El sector residencial representa una porción significativa del consumo total de energía eléctrica, sin embargo, la mayoría de los usuarios carecen de herramientas que les permitan comprender y gestionar su consumo de manera efectiva. La facturación eléctrica tradicional ofrece únicamente un resumen mensual, funcionando como un indicador tardío que no permite identificar patrones de uso específicos ni detectar fuentes de desperdicio, como el conocido "consumo fantasma" o en modo de espera (*standby*) \cite{iea2022standby}.

La problemática central que se aborda es la falta de acceso a información granular y en tiempo real sobre el consumo de energía en el hogar, así como la ausencia de mecanismos accesibles para actuar sobre dicho consumo en caso de anomalías o necesidades de gestión remota. Esta carencia impide a los usuarios tomar decisiones informadas y acciones directas para optimizar el uso de sus aparatos y reducir su gasto energético. Para dar solución a esta necesidad, se propone el diseño y la implementación de un \textbf{sistema inteligente de monitoreo y control energético residencial}, basado en hardware de bajo costo y software de código abierto.

El desarrollo del proyecto se fundamentará en la integración de varias tecnologías clave. En primer lugar, el \textbf{sensado de corriente no invasivo} \cite{fraden2016handbook}, que utiliza un transformador de corriente (SCT-013) para medir el flujo eléctrico de forma segura. En segundo lugar, un \textbf{sistema embebido} (ESP32) \cite{vahid2002embedded} actuará como nodo de adquisición, procesando la señal del sensor y transmitiendo los datos a través del protocolo de mensajería \textbf{MQTT}, estándar en el Internet de las Cosas (IoT) por su eficiencia. Finalmente, se configurará un \textbf{servidor local en una Raspberry Pi} que centralizará la gestión del sistema: el broker \textbf{Mosquitto} recibirá los datos, \textbf{Node-RED} orquestará el flujo de información para su procesamiento, detección de anomalías y envío de \textbf{notificaciones vía Telegram}, una base de datos \textbf{InfluxDB} almacenará las series temporales de consumo, y \textbf{Grafana} proporcionará una interfaz web alojada en la misma \textbf{Raspberry Pi} para la \textbf{visualización} de los datos en tiempo real e históricos.

Este sistema no solo proporcionará una medición precisa del consumo, sino que, a través del análisis de patrones en Node-RED, permitirá la \textbf{detección inteligente de anomalías} y ofrecerá al usuario la capacidad de \textbf{controlar remotamente el suministro eléctrico} principal mediante un \textbf{contactor}, añadiendo una capa de gestión activa y seguridad al monitoreo energético del hogar.
\newpage
\newpage
\newpage % Nueva página

\section{Antecedentes}

    El monitoreo del consumo de energía eléctrica, mediante tecnologías del IoT, ha sido abordado en diversos trabajos académicos a nivel nacional e internacional. Para contextualizar la presente propuesta y destacar su originalidad, se analizan a continuación tres trabajos relevantes.

    \subsection{Trabajo de Grado CUC: Diseño de un prototipo de un sistema domótico para la monitorización del consumo de energía eléctrica}
    Este trabajo, realizado en la Universidad de la Costa (CUC), Colombia, describe el diseño de un sistema domótico que incluye la monitorización del consumo eléctrico en una vivienda. El prototipo utiliza sensores conectados a un microcontrolador y visualiza los datos a través de una aplicación móvil desarrollada para tal fin, permitiendo al usuario observar el consumo de diferentes electrodomésticos \cite{AhumadaBarros2021}.

    \paragraph{Semejanzas:} Comparte con nuestra propuesta el enfoque en el sector residencial y el objetivo de proporcionar al usuario información sobre su consumo eléctrico a través de una interfaz accesible. Ambos proyectos reconocen la importancia de la monitorización para la gestión energética en el hogar.

    \paragraph{Diferencias y Aportación del Proyecto:} Las diferencias clave residen en la arquitectura del sistema y el alcance funcional. Mientras que el trabajo de Ahumada y Barros se centra en la visualización a través de una app móvil, posiblemente conectada a una nube o a un sistema domótico específico, nuestra propuesta implementa una arquitectura de procesamiento local robusta basada en una Raspberry Pi. La aportación distintiva de nuestro proyecto es la integración de herramientas de código abierto como Node-RED, InfluxDB y Grafana en un servidor local, lo que permite no solo la visualización sino también el almacenamiento eficiente de series temporales, el procesamiento avanzado de datos, la detección de anomalías y el envío de notificaciones proactivas a través de Telegram. Adicionalmente, nuestro sistema incorpora una capacidad de control activo, permitiendo la interrupción remota del suministro eléctrico mediante un contactor, funcionalidad orientada a la seguridad y gestión que no se detalla en el antecedente.

    \subsection{Trabajo Final UNLP: Diseño e Implementación de un Sistema de Medición y Visualización del Consumo Eléctrico Residencial}
    Este trabajo final de la Universidad Nacional de La Plata (UNLP), Argentina, detalla el desarrollo de un sistema para medir y visualizar el consumo eléctrico residencial utilizando hardware de código abierto y sensores de corriente. Los datos son procesados y presentados al usuario a través de una interfaz web local o una plataforma de visualización \cite{Cicutti2019}.

    \paragraph{Semejanzas:} Al igual que nuestra propuesta, este proyecto emplea hardware accesible (posiblemente microcontroladores y sensores similares) y busca ofrecer una herramienta de visualización del consumo para usuarios residenciales como alternativa económica.

    \paragraph{Diferencias y Aportación del Proyecto:} Aunque ambos proyectos pueden utilizar interfaces web locales, nuestra propuesta se distingue por la orquestación explícita del flujo de datos mediante Node-RED, el almacenamiento en una base de datos de series temporales como InfluxDB y la visualización especializada con Grafana, herramientas estándar en entornos industriales y de IoT. La aportación central de nuestro trabajo no es solo la medición, sino la creación de un sistema inteligente local capaz de analizar patrones, generar alertas contextualizadas vía Telegram y ejecutar acciones de control sobre el suministro eléctrico, conformando un ecosistema de gestión energética más completo y proactivo que una simple interfaz de visualización.

    \subsection{Proyecto de Código Abierto: OpenEnergyMonitor}
    OpenEnergyMonitor es un proyecto internacional de hardware y software de código abierto para el monitoreo de energía. Su plataforma permite medir consumo, generación y controlar cargas, utilizando hardware modular y su software EmonCMS, una potente plataforma web auto-alojable \cite{openenergymonitor}.

    \paragraph{Semejanzas:} Comparte la filosofía de código abierto, el uso de sensores no invasivos y el objetivo de proporcionar herramientas para entender el uso de la energía. La capacidad de auto-alojamiento de EmonCMS es conceptualmente similar a nuestro enfoque de servidor local.

    \paragraph{Diferencias y Aportación del Proyecto:} OpenEnergyMonitor, con EmonCMS, ofrece una solución muy completa pero también compleja, orientada a usuarios con conocimientos técnicos avanzados. La aportación de nuestra propuesta radica en utilizar una combinación diferente y muy popular de herramientas de código abierto (MQTT, Node-RED, InfluxDB, Grafana) que, si bien puede no tener todas las funcionalidades de EmonCMS, ofrece una curva de aprendizaje más accesible y una gran flexibilidad para la integración con otros sistemas domóticos o de notificación (como Telegram). Nuestro enfoque se centra en construir un sistema de monitoreo y control inteligente, funcional y fácil de replicar, utilizando un stack de software ampliamente adoptado en la comunidad IoT actual.

    \begin{table}[H] % The [H] option (requires 'float' package) tries to place the table exactly here
        \centering

        \small % Use smaller font size if needed to fit
        \begin{tabular}{>{\centering\arraybackslash}p{1.5cm} p{6.5cm} p{6.5cm}} % Adjust p{} widths as needed
            \toprule
            \textbf{Ref.} & \textbf{Similitudes} & \textbf{Diferencias} \\
            \midrule

            \cite{AhumadaBarros2021} & 
            \begin{itemize}[leftmargin=*, topsep=0pt, itemsep=0pt, parsep=0pt, partopsep=0pt]
                \item Enfoque en sector residencial.
                \item Proporciona información de consumo al usuario.
                \item Utiliza interfaz gráfica (App móvil vs. Web/Grafana).
            \end{itemize} & 
            \begin{itemize}[leftmargin=*, topsep=0pt, itemsep=0pt, parsep=0pt, partopsep=0pt]
                \item Medición individual vs. Medición total no invasiva (SCT-013).
                \item Arquitectura: No especificada vs. Servidor local definido (RPi, MQTT, Node-RED, InfluxDB, Grafana).
                \item Funcionalidad: Solo visualización vs. Visualización + Análisis + Alertas (Telegram) + Control (Contactor).
            \end{itemize} \\
            
            \midrule

            \cite{Cicutti2019} & 
            \begin{itemize}[leftmargin=*, topsep=0pt, itemsep=0pt, parsep=0pt, partopsep=0pt]
                \item Enfoque residencial.
                \item Uso de hardware accesible (código abierto).
                \item Ofrece visualización de consumo (posiblemente local).
            \end{itemize} & 
            \begin{itemize}[leftmargin=*, topsep=0pt, itemsep=0pt, parsep=0pt, partopsep=0pt]
                \item Arquitectura: No detallada vs. Servidor local completo (RPi, MQTT, Node-RED, InfluxDB, Grafana).
                \item Funcionalidad: Principalmente visualización vs. Visualización + Análisis + Alertas + Control.
                \item Software: No especificado vs. Herramientas estándar y flexibles.
            \end{itemize} \\
            
            \midrule

            \cite{openenergymonitor} & 
            \begin{itemize}[leftmargin=*, topsep=0pt, itemsep=0pt, parsep=0pt, partopsep=0pt]
                \item Filosofía de código abierto.
                \item Uso de sensado no invasivo.
                \item Opción de auto-alojamiento (EmonCMS vs. RPi stack).
                \item Objetivo de entender el uso de energía.
            \end{itemize} & 
            \begin{itemize}[leftmargin=*, topsep=0pt, itemsep=0pt, parsep=0pt, partopsep=0pt]
                \item Complejidad: Nivel profesional, modular vs. Simplificado, integrado.
                \item Software: EmonCMS vs. Stack MQTT/Node-RED/InfluxDB/Grafana.
                \item Usuario objetivo: Técnico avanzado vs. Usuario no técnico.
                \item Funcionalidad: Medición/Generación vs. Monitoreo + Control básico + Notificaciones (Telegram).
            \end{itemize} \\

            \bottomrule
        \end{tabular}
        \caption{Comparación cualitativa de los trabajos relacionados con el proyecto.}
        \label{tab:comparacion_antecedentes}
    \end{table}
\newpage % Nueva página
    
\section{Justificación}
    La gestión eficiente de la energía eléctrica en el sector residencial constituye un desafío significativo, originado principalmente por la falta de acceso a información detallada y en tiempo real sobre el consumo, así como por la ausencia de mecanismos accesibles para actuar remotamente sobre el suministro. El problema central afecta a los usuarios domésticos, quienes reciben una facturación mensual insuficiente para comprender sus patrones de uso, identificar aparatos ineficientes o cuantificar el consumo fantasma \cite{sap2022smartmetering}. Esta carencia de datos y control representa una barrera para implementar estrategias efectivas de ahorro y gestión energética.

    La solución propuesta es el desarrollo de un sistema de gestión energética inteligente que resuelve esta problemática mediante la ingeniería electrónica. Su importancia radica en empoderar al usuario con datos accionables y herramientas de control, impactando directamente en la reducción del costo eléctrico y fomentando la sostenibilidad \cite{scielo2024iot}. La aportación técnica es la integración de un stack de IoT (ESP32, MQTT, Raspberry Pi, Node-RED, InfluxDB, Grafana) con un actuador de potencia (contactor). El proyecto es factible y viable, ya que se basa en componentes de bajo costo (ESP32, SCT-013) \cite{programarfacil2017sct013} y software de código abierto, consistentes con las competencias de la licenciatura y el tiempo disponible para el Proyecto de Integración. Los beneficiarios directos son los usuarios residenciales, quienes obtienen una herramienta de control y seguridad, y el prototipo podrá servir como material didáctico en los laboratorios del departamento.
    \newpage

\section{Objetivos}

    \subsection{Objetivo General}
        Diseñar e Implementar un sistema electrónico inteligente para el monitoreo y control del consumo energético residencial, para proporcionar al usuario una herramienta accesible de gestión remota del suministro eléctrico.

\subsection{Objetivos Particulares}
    \begin{enumerate}
        \item \textbf{Diseñar} el circuito electrónico de acondicionamiento para la señal del sensor de corriente no invasivo SCT-013, asegurando que la salida de voltaje se mantenga dentro del rango operativo de 0V a 3.3V del convertidor analógico-digital (ADC) del ESP32.
        \item \textbf{Construir} el nodo de medición, ensamblando el sensor SCT-013, el circuito de acondicionamiento diseñado en el objetivo anterior y el microcontrolador ESP32.
        \item \textbf{Diseñar e implementar} el firmware para el ESP32 que adquiera las lecturas del sensor, calcule la potencia eléctrica instantánea y publique los resultados vía MQTT al servidor local.
        \item \textbf{Establecer} el servidor local en la Raspberry Pi, instalando y configurando el broker MQTT, la base de datos InfluxDB y el sistema de visualización Grafana.
        \item \textbf{Programar} en Node-RED los flujos de trabajo para procesar los datos MQTT, almacenarlos en InfluxDB, detectar condiciones de consumo anómalo y activar las notificaciones o el control del contactor.
        \item \textbf{Implementar} el sistema de notificación de alertas por consumo anómalo mediante un bot de Telegram controlado desde Node-RED.
        \item \textbf{Integrar} el control del contactor eléctrico al sistema, permitiendo su activación remota segura desde Node-RED a través de un módulo relay.
        \item \textbf{Verificar} que la precisión del sistema de medición se encuentre dentro de un margen de error del $\pm 5\%$ en comparación con un wattmetro comercial, bajo diferentes perfiles de carga residencial.
        \item \textbf{Validar} la funcionalidad completa del sistema, verificando la correcta operación de la visualización en Grafana, las notificaciones en Telegram y la activación del contactor.
    \end{enumerate}
\newpage

\section{Metodología - Descripción Técnica}

    El presente proyecto consiste en el diseño e implementación de un sistema electrónico para el monitoreo y control del consumo eléctrico residencial. El sistema se compone de un nodo de adquisición de datos basado en un microcontrolador ESP32 y un servidor local basado en una Raspberry Pi, el cual centraliza el procesamiento, almacenamiento, visualización y control del sistema. La comunicación entre el nodo de adquisición y el servidor se realiza mediante el protocolo de mensajería MQTT a través de una red WiFi local.

    \subsection{Diagrama a Bloques del Sistema}
        La arquitectura del sistema propuesto se divide en tres subsistemas funcionales principales, como se ilustra en el diagrama a bloques de la Figura \ref{fig:diagrama_bloques}. Estos subsistemas son: 1) El Nodo de Medición y Adquisición, 2) El Servidor Local de Procesamiento, y 3) El Subsistema de Control y Notificación.
        \begin{figure}[H]
            \centering
            \includegraphics[width=0.8\textwidth]{img/diagrama_bloques.png}
            \caption{Diagrama a bloques funcional del sistema propuesto.}
            \label{fig:diagrama_bloques}
        \end{figure}

         \newpage

    \subsection{Descripción de Subsistemas}
        \paragraph{1. Subsistema de Medición (Nodo ESP32):}
        Es el componente encargado de la adquisición de datos.
        \begin{itemize}
            \item \textbf{Sensor (SCT-013):} Es un transformador de corriente no invasivo de núcleo partido. Su función es medir la corriente alterna (AC) que fluye por el conductor principal de la vivienda sin contacto eléctrico.
                \begin{itemize}
                    \item \textit{Entrada:} Corriente AC (0-100A) de la línea de fase.
                    \item \textit{Salida:} Una señal de corriente AC pequeña, proporcional a la entrada (ej. 50mA).
                \end{itemize}
            \item \textbf{Acondicionamiento de Señal:} Este circuito (basado en resistencias y capacitores) convierte la señal de corriente AC del sensor en una señal de voltaje DC con un desfase (offset), adaptada para ser leída por el ESP32 (ej. 0-3.3V).
                \begin{itemize}
                    \item \textit{Entrada:} Señal de corriente AC del sensor.
                    \item \textit{Salida:} Señal de voltaje DC (0-3.3V) al ADC del ESP32.
                \end{itemize}
            \item \textbf{Procesamiento (ESP32):} El microcontrolador ESP32 muestrea la señal del ADC, realiza los cálculos para obtener la Potencia instantánea (W) y la Potencia aparente (VA), y se conecta a la red WiFi local.
                \begin{itemize}
                    \item \textit{Entrada:} Señal de voltaje del acondicionador.
                    \item \textit{Salida:} Paquetes de datos MQTT (Protocolo: MQTT v3.1.1).
                \end{itemize}
        \end{itemize}

        \paragraph{2. Subsistema Servidor Local (Raspberry Pi):}
        Es el cerebro del sistema. Centraliza la recepción, procesamiento y almacenamiento de datos.
        \begin{itemize}
            \item \textbf{Broker MQTT:} Se utilizará un broker (ej. Mosquitto) que actúa como intermediario de mensajería. Recibe los datos publicados por el ESP32.
            \item \textbf{Node-RED:} Es la herramienta de programación visual que orquesta los flujos de datos. Se suscribe al broker MQTT, recibe los datos de potencia, los procesa (ej. calcula costos, detecta anomalías) y los distribuye a los demás servicios (InfluxDB, Telegram, Contactor).
            \item \textbf{InfluxDB:} Es la base de datos optimizada para series temporales. Almacena eficientemente los datos de consumo (Watts, Amperes, Costo) con su marca de tiempo para consultas históricas.
            \item \textbf{Grafana:} Es la plataforma de visualización. Se conecta a InfluxDB como fuente de datos y permite crear un dashboard web con gráficas, medidores e indicadores del consumo en tiempo real e histórico.
        \end{itemize}

        \paragraph{3. Subsistema de Control y Notificación:}
        Son los componentes que interactúan con el usuario o con la instalación eléctrica.
        \begin{itemize}
            \item \textbf{Telegram Bot:} La integración se realiza directamente desde el servidor local utilizando la API de Bots de Telegram (mediante nodos específicos en Node-RED). La Raspberry Pi procesa los datos localmente y decide cuándo enviar una alerta; de esta manera, la conexión a internet se utiliza únicamente para transmitir mensajes puntuales de notificación o recibir comandos (ej. /cortar\_suministro), asegurando que el registro histórico de consumo no se almacene en nubes de terceros.
            \item \textbf{Contactor Eléctrico:} Es un actuador de potencia controlado por la Raspberry Pi (a través de un módulo relay). Al recibir una señal de Node-RED (originada por una alerta o un comando del usuario), este dispositivo es capaz de interrumpir físicamente el suministro eléctrico principal de la vivienda como medida de seguridad.
        \end{itemize}

    \subsection{Especificaciones Técnicas}
        El prototipo a desarrollar buscará cumplir con las siguientes especificaciones:
        \begin{itemize}
            \item \textbf{Rango de Medición:} 0 – 100 Amperes (AC).
            \item \textbf{Voltaje de Operación (Medición):} 127V AC (Monofásico).
            \item \textbf{Precisión Esperada (Potencia):} $\pm 5\%$ comparado con un medidor comercial.
            \item \textbf{Frecuencia de Muestreo (ESP32):} Envío de datos al servidor cada 10 segundos.
            \item \textbf{Plataforma (Servidor):} Raspberry Pi 3B+ o superior.
            \item \textbf{Protocolo de Comunicación:} MQTT sobre WiFi (red local).
            \item \textbf{Interfaz de Usuario:} Dashboard en Grafana accesible vía navegador web en la red local.
            \item \textbf{Sistema de Control:} Activación de contactor de 2 Polos (60A) mediante relay controlado por GPIO de la Raspberry Pi.
        \end{itemize}
\newpage
    
\section{Cronograma de Actividades}

    La planificación del proyecto se ha estructurado para ser completada en dos trimestres consecutivos (24 semanas), cubriendo las UEA del Trimestre 26-I y 26-P. Las actividades están directamente alineadas con los objetivos particulares definidos en la sección anterior. El seguimiento del avance se realizará mediante la siguiente tabla (Ver Tabla \ref{tab:cronograma}).

    \begin{table}[H]
    \centering
    \caption{Cronograma de Actividades del Proyecto.}
    \label{tab:cronograma}
    \resizebox{\textwidth}{!}{%
    \begin{tabular}{|l|c|c|c|c|c|c|c|c|c|c|c|c|c|c|c|c|c|c|c|c|c|c|c|c|}
    \hline
    & \multicolumn{12}{c|}{\textbf{Trimestre 26-I}} & \multicolumn{12}{c|}{\textbf{Trimestre 26-P}} \\ \hline
    \textbf{Actividad} & \textbf{1} & \textbf{2}         & \textbf{3} & \textbf{4} & \textbf{5} & \textbf{6} & \textbf{7} & \textbf{8} & \textbf{9} & \textbf{10} & \textbf{11} & \textbf{12} & \textbf{1} & \textbf{2} & \textbf{3} & \textbf{4} & \textbf{5} & \textbf{6} & \textbf{7} & \textbf{8} & \textbf{9} & \textbf{10} & \textbf{11} & \textbf{12} \\ \hline
    \footnotesize{1. Diseño del circuito (Obj. 1)}       & \cellcolor{gray!40} & \cellcolor{gray!40} & \cellcolor{gray!40} & & & & & & & & & & & & & & & & & & & & & \\ \hline
    \footnotesize{2. Construcción prototipo (Obj. 2)}    & & & \cellcolor{gray!40} & \cellcolor{gray!40} & \cellcolor{gray!40} & \cellcolor{gray!40} & & & & & & & & & & & & & & & & & & \\ \hline
    \footnotesize{3. Desarrollo firmware ESP32 (Obj. 3)} & & & & & & \cellcolor{gray!40} & \cellcolor{gray!40} & \cellcolor{gray!40} & \cellcolor{gray!40} & & & & & & & & & & & & & & & \\ \hline
    \footnotesize{4. Establ. servidor RPi (Obj. 4)}      & & & & & & & & & \cellcolor{gray!40} & \cellcolor{gray!40} & \cellcolor{gray!40} & \cellcolor{gray!40} & & & & & & & & & & & & \\ \hline
    \footnotesize{5. Programación Node-RED (Obj. 5)}     & & & & & & & & & & & & \cellcolor{gray!40} & \cellcolor{gray!40} & \cellcolor{gray!40} & \cellcolor{gray!40} & & & & & & & & & \\ \hline
    \footnotesize{6. Implementar Telegram (Obj. 6)}      & & & & & & & & & & & & & & & \cellcolor{gray!40} & \cellcolor{gray!40} & \cellcolor{gray!40} & \cellcolor{gray!40} & & & & & & \\ \hline
    \footnotesize{7. Integrar contactor (Obj. 7)}        & & & & & & & & & & & & & & & & & \cellcolor{gray!40} & \cellcolor{gray!40} & \cellcolor{gray!40} & & & & & \\ \hline
    \footnotesize{8. Evaluación precisión (Obj. 8)}      & & & & & & & & & & & & & & & & & & & \cellcolor{gray!40} & \cellcolor{gray!40} & \cellcolor{gray!40} & & & \\ \hline
    \footnotesize{9. Validación funcional (Obj. 9)}      & & & & & & & & & & & & & & & & & & & & & \cellcolor{gray!40} & \cellcolor{gray!40} & & \\ \hline
    \footnotesize{Redacción de informe}                  & & \cellcolor{gray!40} & \cellcolor{gray!40} & \cellcolor{gray!40} & \cellcolor{gray!40} & \cellcolor{gray!40} & \cellcolor{gray!40} & \cellcolor{gray!40} & \cellcolor{gray!40} & \cellcolor{gray!40} & \cellcolor{gray!40} & \cellcolor{gray!40} & \cellcolor{gray!40} & \cellcolor{gray!40} & \cellcolor{gray!40} & \cellcolor{gray!40} & \cellcolor{gray!40} & \cellcolor{gray!40} & \cellcolor{gray!40} & \cellcolor{gray!40} & \cellcolor{gray!40} & \cellcolor{gray!40} & \cellcolor{gray!40} & \cellcolor{gray!40} \\ \hline
    \footnotesize{Entrega y presentación}                & & & & & & & & & & & & & & & & & & & & & & & & \cellcolor{gray!40} \\ \hline
    \end{tabular}%
    }
    \end{table}
    

    % --- SECCIÓN 7: ENTREGABLES ---
\section{Entregables}
    Para el desarrollo del proyecto y como evidencia del cumplimiento de los objetivos, se generarán los siguientes productos:

    \begin{itemize}
        \item \textbf{Documentación de Diseño:} Diagramas esquemáticos del circuito de acondicionamiento de señal y diagrama de conexión del sistema de control (contactor).
        \item \textbf{Prototipo Físico Funcional:} Nodo de medición (ESP32 y circuito de acondicionamiento) y el servidor local (Raspberry Pi) configurados e integrados.
        \item \textbf{Código Fuente (Software):}
            \begin{itemize}
                \item Firmware del microcontrolador ESP32 (programado en C++/Arduino).
                \item Flujos de trabajo de Node-RED (en formato JSON).
            \end{itemize}
        \item \textbf{Panel de Control (Dashboard):} Dashboard funcional y configurado en Grafana para la visualización de los datos de consumo.

    \end{itemize}
    \newpage

 % Comentarios de código para posibles figuras o tablas futuras
\begin{comment}
 %imagenes
    \begin{figure}[H]
        \centering
        \includegraphics[width=0.8\textwidth]{img/PPG.png}
        \caption[Diagrama de la fotopletismografía]{Diagrama de la fotopletismografía\footnotemark}
        \label{fig:PPG}
    \end{figure}
    \footnotetext{Diagrama de la fotopletismografía. Imagen tomada de Blog da robotica. Fuente: \url{https://www.blogdarobotica.com/}}
% tablas
    \begin{table}[t]
        \begin{center}
        \begin{tabular}{ l | c | c}
            Aproximación & Factor de calidad Q & Constante K\\ \hline
            Butterworth & 0.7071 & 1.0000\\
            Chebyshev (cresta de 0.01db) & 0.7247 & 1.0231\\
            Chebyshev (cresta de 0.1db) & 0.7673 & 1.0674 \\
            Chebyshev (cresta de 0.25db) &  0.8093 & 1.0991 \\
            Chebyshev (cresta de 0.5db) & 0.8638 & 1.1286 \\
            Chebyshev (cresta de 1db) & 0.9564 & 1.1596 \\
            Bessel & 0.5771 & 0.7840 \\
            \end{tabular}
        \caption{Constantes de los filtros}
        \label{tab:constantes}
        \end{center}
    \end{table} 
\end{comment}